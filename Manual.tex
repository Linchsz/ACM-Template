
\documentclass[twoside]{article}
\usepackage[colorlinks,linkcolor=black]{hyperref}
\usepackage{xeCJK}
\usepackage{fancyhdr}
\usepackage{amsmath, amsthm}
\usepackage{listings,xcolor}
\usepackage{geometry}
\usepackage{fontspec}
\setsansfont{Monaco}
\setmonofont[Mapping={}]{Monaco}
\newcommand{\HRule}{\rule{\linewidth}{0.5mm}}
\geometry{left=2.5cm,right=2.5cm,top=2.5cm,bottom=2.5cm}
\lstset{
    language    = c++,
    breaklines  = true,
    captionpos  = b,
    tabsize     = 4,
    numbers     = left,
    columns     = fullflexible,
    keepspaces  = true,
    commentstyle = \color[RGB]{0,128,0},
    keywordstyle = \color[RGB]{0,0,255},
    basicstyle   = \small\ttfamily,
    rulesepcolor = \color{red!20!green!20!blue!20},
    showstringspaces = false,
}
\title{ICPC Template Manual}
\author{Star}
\pagestyle{fancy}
\fancyhf{}
\fancyhead[C]{ICPC Template Manual, Star}
\begin{document}\small
\HRule \\ [1cm]
\textbf{\Huge{ICPC Template Manual}} \\ [0.5cm]
\HRule \\ [4cm]
\textbf{\Huge{University of Jinan}} \\ [1cm]
\LARGE{Star}
\vfill
\Large{\today}
\end{center}
\clearpage
\end{titlepage}
\tableofcontents\clearpage
\pagestyle{fancy}
\lfoot{}
\cfoot{\thepage}\rfoot{}
\setcounter{section}{-1}
\setcounter{page}{1}
\clearpage\section{Header}
\begin{lstlisting}
#include <bits/stdc++.h>
using namespace std;

const double eps = 1e-6;
const int mod = 1e9 + 7;
const int INF = 0x3f3f3f3f;
const double pi = 4.0 * atan(1.0);

typedef long long ll;
typedef long double ld;
typedef vector<ll> vl;
typedef vector<int> vi;
typedef pair<int, int> pii;

#define fi first
#define se second
#define gc getchar()
#define pc(x) putchar(x)
#define pb(x) push_back(x)
#define eb(x) emplace_back(x)
#define rd_() rd<__int128>()
#define wd_(x) wr<__int128>(x)
#define print(x, c) wr(x), putchar(c)
#define rep(i, n) for (int i = 0; i < (n); ++i)
#define repn(i, n) for (int i = 1; i <= (n); ++i)

template <typename T>
inline T rd() {
    T x = 0, f = 1;
    char c = getchar();
    while (!isdigit(c)) f = c == '-' ? -1 : 1, c = getchar();
    while (isdigit(c)) x = (x << 1) + (x << 3) + (c ^ 48), c = getchar();
    return x * f;
}

template <typename T>
inline void wr(T x) {
    T y = 1, len = 1;
    if (x < 0) x = -x, putchar('-');
    while (y <= x / 10) y = (y << 1) + (y << 3), ++len;
    for (; len; --len) putchar(x / y ^ 48), x %= y, y /= 10;
}

int main() {
#ifdef IO
    freopen("test.in", "r", stdin);
    freopen("test.out", "w", stdout);
#endif

    return 0;
}
\end{lstlisting}
\clearpage\section{Math}
\subsection{Prime}
\subsubsection{Eratosthenes Sieve}
\begin{lstlisting}
\end{lstlisting}
$O(n\log\log n)$筛出maxn内所有素数\\
$notprime[i] = 0/1$ 0为素数 1为非素数\\
\begin{lstlisting}
const int maxn = "Edit";
bool notprime[maxn] = {1, 1};   // 0 && 1 为非素数
void GetPrime()
{
    for (int i = 2; i < maxn; i++)
        if (!notprime[i] && i <= maxn / i)  // 筛到√n为止
            for (int j = i * i; j < maxn; j += i)
                notprime[j] = 1;
}
\end{lstlisting}
\subsubsection{Eular Sieve}
\begin{lstlisting}
\end{lstlisting}
$O(n)$得到欧拉函数$phi[]$、素数表$prime[]$、素数个数$tot$\\
\begin{lstlisting}
const int maxn = "Edit";
bool vis[maxn];
int tot, phi[maxn], prime[maxn];
void CalPhi()
{
    phi[1] = 1;
    for (int i = 2; i < maxn; i++)
    {
        if (!vis[i])
            prime[tot++] = i, phi[i] = i - 1;
        for (int j = 0; j < tot; j++)
        {
            if (i * prime[j] > maxn) break;
            vis[i * prime[j]] = 1;
            if (i % prime[j] == 0)
            {
                phi[i * prime[j]] = phi[i] * prime[j];
                break;
            }
            else
                phi[i * prime[j]] = phi[i] * (prime[j] - 1);
        }
    }
}
\end{lstlisting}
$d(n)$函数
\begin{lstlisting}
const int maxn = "Edit";
int prime[maxn], tot;
int d[maxn], e[maxn]; //d正除数个数,e最小质因子个数
bool check[maxn];
void CalD()
{
    d[1] = 1;
    for (int i = 2; i < maxn; i++)
    {
        if (!check[i])
        {
            prime[tot++] = i;
            e[i] = 1, d[i] = 2; 
        }
        for (int j = 0; j < tot; j++)
        {
            if (i * prime[j] >= maxn) break;
            check[i * prime[j]] = true;
            if (i % prime[j] == 0)
            {
                e[i * prime[j]] = e[i] + 1;
                d[i * prime[j]] = d[i] / e[i] * (e[i] + 1);
                break;
            }
            else
            {
                e[i * prime[j]] = 1;
                d[i * prime[j]] = 2 * d[i];
            }
        }
    }
}
\end{lstlisting}
$\sigma \lambda(n)$函数,$\lambda=1$
\begin{lstlisting}
const int maxn = "Edit";
int prime[maxn], tot;
int sig[maxn], e[maxn]; //sig正除数,e不含能整除i的最小质因子的正除数和
bool check[maxn];
void CalSig()
{
    sig[1] = 1;
    for (int i = 2; i < maxn; i++)
    {
        if (!check[i])
        {
            prime[tot++] = i;
            e[i] = 1, sig[i] = i + 1;
        }
        for (int j = 0; j < tot; j++)
        {
            if (i * prime[j] >= maxn) break;
            check[i * prime[j]] = true;
            if (i % prime[j] == 0)
            {
                sig[i * prime[j]] = sig[i] * prime[j] + e[i];
                e[i * prime[j]] = e[i];
                break;
            }
            else
            {
                sig[i * prime[j]] = sig[i] * (prime[j] + 1);
                e[i * prime[j]] = sig[i];
            }
        }
    }
}\end{lstlisting}
\subsubsection{Prime Factorization}
\begin{lstlisting}
vector<pair<ll, int>> getFactors(ll x)
{
    vector<pair<ll, int>> fact;
    for (int i = 0; prime[i] <= x / prime[i]; i++)
    {
        if (x % prime[i] == 0)
        {
            fact.emplace_back(prime[i], 0);
            while (x % prime[i] == 0) fact.back().second++, x /= prime[i];
        }
    }
    if (x != 1) fact.emplace_back(x, 1);
    return fact;
}\end{lstlisting}
\subsubsection{Miller Rabin}
\begin{lstlisting}
\end{lstlisting}
$O(s\log n)$内判定$2^{63}$内的数是不是素数,$s$为测定次数
\begin{lstlisting}
bool Miller_Rabin(ll n, int s)
{
    if (n == 2) return 1;
    if (n < 2 || !(n & 1)) return 0;
    int t = 0;
    ll  x, y, u = n - 1;
    while ((u & 1) == 0) t++, u >>= 1;
    for (int i = 0; i < s; i++)
    {
        ll a = rand() % (n - 1) + 1;
        ll x = Pow(a, u, n);
        for (int j = 0; j < t; j++)
        {
            ll y = Mul(x, x, n);
            if (y == 1 && x != 1 && x != n - 1) return 0;
            x = y;
        }
        if (x != 1) return 0;
    }
    return 1;
}
\end{lstlisting}
\subsubsection{Segment Sieve}
\begin{lstlisting}
\end{lstlisting}
对区间$[a,b)$内的整数执行筛法。\\
函数返回区间内素数个数\\
\texttt{is\_prime[i-a]=true}表示$i$是素数\\
$1<a<b \le 10^{12}, b-a \le 10^6$
\begin{lstlisting}
const int maxn = "Edit";
bool is_prime_small[maxn], is_prime[maxn];
ll prime[maxn];
int segment_sieve(ll a, ll b)
{
    int tot = 0;
    for (ll i = 0; i * i < b; ++i) is_prime_small[i] = true;
    for (ll i = 0; i < b - a; ++i) is_prime[i] = true;
    for (ll i = 2; i * i < b; ++i)
        if (is_prime_small[i])
        {
            for (ll j = 2 * i; j * j < b; j += i)
                is_prime_small[j] = false;
            for (ll j = max(2LL, (a + i - 1) / i) * i; j < b; j += i)
                is_prime[j - a] = false;
        }
    for (ll i = 0; i < b - a; ++i)
        if (is_prime[i]) prime[tot++] = i + a;
    return tot;
}
\end{lstlisting}
\subsection{Euler phi}
\subsubsection{Euler}
\begin{lstlisting}
ll euler(ll n)
{
    ll rt = n;
    for (int i = 2; i * i <= n; i++)
        if (n % i == 0)
        {
            rt -= rt / i;
            while (n % i == 0) n /= i;
        }
    if (n > 1) rt -= rt / n;
    return rt;
}
\end{lstlisting}
\subsubsection{Sieve}
\begin{lstlisting}
const int N = "Edit";
int phi[N] = {0, 1};
void caleuler()
{
    for (int i = 2; i < N; i++)
        if (!phi[i])
            for (int j = i; j < N; j += i)
            {
                if (!phi[j]) phi[j] = j;
                phi[j] = phi[j] / i * (i - 1);
            }
}
\end{lstlisting}
\subsection{Basic Number Theory}
\subsubsection{Extended Euclidean}
\begin{lstlisting}
ll exgcd(ll a, ll b, ll &x, ll &y)
{
    ll d = a;
    if (b) d = exgcd(b, a % b, y, x), y -= x * (a / b);
    else x = 1, y = 0;
    return d;
}
\end{lstlisting}
\subsubsection{ax+by=c}
\begin{lstlisting}
\end{lstlisting}
引用返回通解: $X = x + k * dx, Y = y – k * dy$\\
引用返回的x是最小非负整数解,方程无解函数返回0
\begin{lstlisting}
#define Mod(a, b) (((a) % (b) + (b)) % (b))
bool solve(ll a, ll b, ll c, ll& x, ll& y, ll& dx, ll& dy)
{
    if (a == 0 && b == 0) return 0;
    ll x0, y0;
    ll d = exgcd(a, b, x0, y0);
    if (c % d != 0) return 0;
    dx = b / d, dy = a / d;
    x = Mod(x0 * c / d, dx);
    y = (c - a * x) / b;
    //  y = Mod(y0 * c / d, dy); x = (c - b * y) / a;
    return 1;
}
\end{lstlisting}
\subsubsection{Multiplicative Inverse Modulo}
\begin{lstlisting}
\end{lstlisting}
利用exgcd求$a$在模$m$下的逆元,需要保证$\gcd(a, m) == 1$.
\begin{lstlisting}
ll inv(ll a, ll m)
{
    ll x, y;
    ll d = exgcd(a, m, x, y);
    return d == 1 ? (x + m) % m : -1;
}
\end{lstlisting}
$a < p$ 且 $p$为素数时,有以下两种求法\\
费马小定理
\begin{lstlisting}
ll inv(ll a, ll p) { return Pow(a, p - 2, p); }
\end{lstlisting}
贾志鹏线性筛
\begin{lstlisting}
for (int i = 2; i < n; i++) inv[i] = inv[p % i] * (p - p / i) % p;
\end{lstlisting}
\subsubsection{Discrete Logarithm}
\begin{lstlisting}
\end{lstlisting}
求解$a ^ x \equiv b \pmod p$,$p$可以不是质数
\begin{lstlisting}
ll exbsgs(ll a, ll b, ll p)
{
    if (b == 1LL) return 0;
    ll t, d = 1, k = 0;
    while ((t = gcd(a, p)) != 1)
    {
        if (b % t) return -1;
        ++k, b /= t, p /= t, d = d * (a / t) % p;
        if (b == d) return k;
    }
    map<ll, ll> dic;
    ll m = ceil(sqrt(p));
    ll a_m = Pow(a, m, p), mul = b;
    for (ll j = 1; j <= m; ++j) mul = mul * a % p, dic[mul] = j;
    for (ll i = 1; i <= m; ++i)
    {
        d = d * a_m % p;
        if (dic[d]) return i * m - dic[d] + k;
    }
    return -1;
}\end{lstlisting}
\subsection{Modulo Linear Equation}
\subsubsection{Chinese Remainder Theory}
\begin{lstlisting}
\end{lstlisting}
$X \equiv r_i \pmod {m_i}$; 要求$m_i$两两互质\\
引用返回通解$X = re + k * mo$
\begin{lstlisting}
void crt(ll r[], ll m[], ll n, ll &re, ll &mo)
{
    mo = 1, re = 0;
    for (int i = 0; i < n; i++) mo *= m[i];
    for (int i = 0; i < n; i++)
    {
        ll x, y,  tm = mo / m[i];
        ll d = exgcd(tm, m[i], x, y);
        re = (re + tm * x * r[i]) % mo;
    }
    re = (re + mo) % mo;
}
\end{lstlisting}
\subsubsection{ExCRT}
\begin{lstlisting}
\end{lstlisting}
$X \equiv r_i \pmod {m_i}$; $m_i$可以不两两互质\\
引用返回通解$X = re + k * mo$; 函数返回是否有解
\begin{lstlisting}
bool excrt(ll r[], ll m[], ll n, ll &re, ll &mo)
{
    ll x, y;
    mo = m[0], re = r[0];
    for (int i = 1; i < n; i++)
    {
        ll d = exgcd(mo, m[i],  x, y);
        if ((r[i] - re) % d != 0) return 0;
        x = (r[i] - re) / d * x % (m[i] / d);
        re += x * mo;
        mo = mo / d * m[i];
        re %= mo;
    }
    re = (re + mo) % mo;
    return 1;
}
\end{lstlisting}
\subsection{Combinatorics}
\subsubsection{Combination}
\begin{lstlisting}
\end{lstlisting}
$0 \leq m \leq n \leq 1000$
\begin{lstlisting}
const int maxn = 1010;
ll C[maxn][maxn];
void CalComb()
{
    C[0][0] = 1;
    for (int i = 1; i < maxn; i++)
    {
        C[i][0] = 1;
        for (int j = 1; j <= i; j++) C[i][j] = (C[i - 1][j - 1] + C[i - 1][j]) % mod;
    }
}

\end{lstlisting}
$0 \leq m \leq n \leq 10^5$, 模$p$为素数
\begin{lstlisting}
const int maxn = 100010;
ll f[maxn];
ll inv[maxn]; // 阶乘的逆元
void CalFact()
{
    f[0] = 1;
    for (int i = 1; i < maxn; i++) f[i] = (f[i - 1] * i) % p;
    inv[maxn - 1] = Pow(f[maxn - 1], p - 2, p);
    for (int i = maxn - 2; ~i; i--) inv[i] = inv[i + 1] * (i + 1) % p;
}
ll C(int n, int m) { return f[n] * inv[m] % p * inv[n - m] % p; }
\end{lstlisting}
\subsubsection{Lucas}
\begin{lstlisting}
\end{lstlisting}
$1 \leq n, m \leq 1000000000, 1 < p < 100000$, p是素数
\begin{lstlisting}
const int maxp = 100010;
ll f[maxn];
ll inv[maxn]; // 阶乘的逆元
void CalFact()
{
    f[0] = 1;
    for (int i = 1; i < maxn; i++) f[i] = (f[i - 1] * i) % p;
    inv[maxn - 1] = Pow(f[maxn - 1], p - 2, p);
    for (int i = maxn - 2; ~i; i--) inv[i] = inv[i + 1] * (i + 1) % p;
}
ll Lucas(ll n, ll m, ll p)
{
    ll ret = 1;
    while (n && m)
    {
        ll a = n % p, b = m % p;
        if (a < b) return 0;
        ret = ret * f[a] % p * inv[b] % p * inv[a - b] % p;
        n /= p, m /= p;
    }
    return ret;
}
\end{lstlisting}
\subsubsection{Big Combination}
\begin{lstlisting}
\end{lstlisting}
$0 \leq n \leq 10^9, 0 \leq m \leq 10^4, 1 \leq k \leq 10^9+7$
\begin{lstlisting}
vector<int> v;
int dp[110];
ll Cal(int l, int r, int k, int dis)
{
    ll res = 1;
    for (int i = l; i <= r; i++)
    {
        int t = i;
        for (int j = 0; j < v.size(); j++)
        {
            int y = v[j];
            while (t % y == 0) dp[j] += dis, t /= y;
        }
        res = res * (ll)t % k;
    }
    return res;
}
ll Comb(int n, int m, int k)
{
    memset(dp, 0, sizeof(dp));
    v.clear();
    int tmp = k;
    for (int i = 2; i * i <= tmp; i++)
        if (tmp % i == 0)
        {
            int num = 0;
            while (tmp % i == 0) tmp /= i, num++;
            v.push_back(i);
        }
    if (tmp != 1) v.push_back(tmp);
    ll ans = Cal(n - m + 1, n, k, 1);
    for (int j = 0; j < v.size(); j++) ans = ans * Pow(v[j], dp[j], k) % k;
    ans = ans * inv(Cal(2, m, k, -1), k) % k;
    return ans;
}
\end{lstlisting}
\subsubsection{Polya}
\begin{lstlisting}
\end{lstlisting}
推论:一共$n$个置换,第$i$个置换的循环节个数为$gcd(i,n)$\\
$N*N$的正方形格子,$c^{n^2}+2c^{\frac{n^2+3}{4}}+c^{\frac{n^2+1}{2}}+2c^{n\frac{n+1}{2}}+2c^{\frac{n(n+1)}{2}}$\\
正六面体,$\frac{m^8+17m^4+6m^2}{24}$
正四面体,$\frac{m^4+11m^2}{12}$\\
\begin{lstlisting}
\end{lstlisting}
长度为$n$的项链串用$c$种颜色染
$\sum_{d|n} \frac{\varphi(n / d) c ^ d}{n}$
\begin{lstlisting}
ll solve(int c, int n)
{
    if (n == 0) return 0;
    ll ans = 0;
    for (int i = 1; i <= n; i++) ans += Pow(c, __gcd(i, n));
    if (n & 1) ans += n * Pow(c, n + 1 >> 1);
    else ans += n / 2 * (1 + c) * Pow(c, n >> 1);
    return ans / n / 2;
}
\end{lstlisting}
每种颜色至少涂多少个,求方案数
\begin{lstlisting}
ll polya(int a)//a为循环节长度
{
    ll dp[65][65] = {0}; //前者为颜色,后者为未填充格子个数
    int tot = 60 / a, limit = 0;
    dp[0][tot] = 1;
    for (int i = 1; i <= n; i++)
    {
        int tmp = (c[i] + a - 1) / a;
        int up2 = tot - limit;
        int up1 = up2 - tmp;           //最多空tot-(limit + tmp)
        for (int j = 0; j <= up1; j++) //最少空0个,即填满
        {
            for (int k = tmp; j + k <= up2; k++) //至少选tmp个,最多选tot - limit -j
                (dp[i][j] += dp[i - 1][j + k] * C[j + k][k]) %= p;
        }
        limit += tmp;
    }
    return dp[n][0];
}
\end{lstlisting}
每种颜色要有多少个,求恰好满足的方案数
\begin{lstlisting}
bool check(int b) //a[i]是每种颜色有多少个,b是循环节长度
{
    for (int i = 0; i < n; i++)
        if (a[i] % b) return false;
    return true;
}
ll solve(int tot, int b) //tot是总数,b是循环节长度
{
    if (!check(b)) return 0;
    ll res = 1, cnt = tot / b; //cnt循环节个数
    for (int i = 0; i < 6; i++)
    {
        res *= C[cnt][a[i] / b];
        cnt -= a[i] / b;
    }
    return res;
}
\end{lstlisting}
\subsection{Fast Power}
\begin{lstlisting}
inline ll Mul(ll a, ll b, ll m)
{
    if (m <= 1000000000)
        return a * b % m;
    else if (m <= 1000000000000ll)
        return (((a * (b >> 20) % m) << 20) + (a * (b & ((1 << 20) - 1)))) % m;
    else
    {
        ll d = (ll)floor(a * (long double)b / m + 0.5);
        ll ret = (a * b - d * m) % m;
        if (ret < 0) ret += m;
        return ret;
    }
}
ll Pow(ll a, ll n, ll m)
{
    ll t = 1;
    for (; n; n >>= 1, a = (a * a % m))
        if (n & 1) t = (t * a % m);
    return t;
}
\end{lstlisting}
\subsection{Mobius Inversion}
\subsubsection{Mobius}
\begin{lstlisting}
\end{lstlisting}
$F(n)=\sum_{d|n}f(d)\Rightarrow f(n)=\sum_{d|n}\mu(d)F(\frac{n}{d})$\\
$F(n)=\sum_{n|d}f(d)\Rightarrow f(n)=\sum_{n|d}\mu(\frac{d}{n})F(d)$
\begin{lstlisting}
const int maxn = "Edit";
int prime[maxn], tot, mu[maxn];
bool check[maxn];
void CalMu()
{
    mu[1] = 1;
    for (int i = 2; i < maxn; i++)
    {
        if (!check[i]) prime[tot++] = i, mu[i] = -1;
        for (int j = 0; j < tot; j++)
        {
            if (i * prime[j] >= maxn) break;
            check[i * prime[j]] = true;
            if (i % prime[j] == 0)
            {
                mu[i * prime[j]] = 0;
                break;
            }
            else
                mu[i * prime[j]] = -mu[i];
        }
    }
}
\end{lstlisting}
\subsubsection{Examples}
\begin{lstlisting}
\end{lstlisting}
有$n$个数$(n \leq 100000,1 \le  a_i \le 10 ^ 6)$,问这$n$个数中互质的数的对数
\begin{lstlisting}
const int maxn = "Edit";
int b[maxn];
ll solve(int n)
{
    ll ans = 0;
    for (int i = 0, x; i < n; i++) scanf("%d", &x), b[x]++;
    for (int i = 1; i < maxn; i++)
    {
        int cnt = 0;
        for (int j = i; j < maxn; j += i) cnt += b[j];
        ans += 1LL * mu[i] * cnt * cnt;
    }
    return (ans - b[1]) / 2;
}
\end{lstlisting}
$\gcd(x,y)=1$的对数, $x \leq n, y \leq m$
\begin{lstlisting}
ll solve(int n, int m)
{
    if (n > m) swap(n, m);
    ll ans = 0;
    for (int i = 1; i <= n; i++) ans += (ll)mu[i] * (n / i) * (m / i);
    /* 
    数论分块写法(sum为莫比乌斯函数的前缀和)
    for (int i = 1; i <= n; i = pos + 1)
    {
        pos = min(n / (n / i), m / (m / i));
        ans += 1LL * (sum[pos] - sum[i - 1]) * (n / i) * (m / i);
    }
    */
    return ans;
}

\end{lstlisting}
\subsection{Fast Transformation}
\subsubsection{FFT}
\begin{lstlisting}
const double PI = acos(-1.0);
//复数结构体
struct Complex
{
    double x, y; //实部和虚部 x+yi
    Complex(double _x = 0.0, double _y = 0.0) { x = _x, y = _y; }
    Complex operator-(const Complex& b) const { return Complex(x - b.x, y - b.y); }
    Complex operator+(const Complex& b) const { return Complex(x + b.x, y + b.y); }
    Complex operator*(const Complex& b) const { return Complex(x * b.x - y * b.y, x * b.y + y * b.x); }
};
void change(Complex y[], int len)
{
    for (int i = 1, j = len / 2; i < len - 1; i++)
    {
        if (i < j) swap(y[i], y[j]);
        int k = len / 2;
        while (j >= k) j -= k, k /= 2;
        if (j < k) j += k;
    }
}
/*
* len必须为2^k形式,
* on==1时是DFT,on==-1时是IDFT
*/
void fft(Complex y[], int len, int on)
{
    change(y, len);
    for (int h = 2; h <= len; h <<= 1)
    {
        Complex wn(cos(-on * 2 * PI / h), sin(-on * 2 * PI / h));
        for (int j = 0; j < len; j += h)
        {
            Complex w(1, 0);
            for (int k = j; k < j + h / 2; k++)
            {
                Complex u = y[k];
                Complex t = w * y[k + h / 2];
                y[k] = u + t, y[k + h / 2] = u - t;
                w = w * wn;
            }
        }
    }
    if (on == -1)
        for (int i = 0; i < len; i++) y[i].x /= len;
}
\end{lstlisting}
\subsubsection{NTT}
\begin{lstlisting}
\end{lstlisting}
模数P为费马素数,G为P的原根。
$G^{\frac{P-1}{n}}$具有和$w_n=e^{\frac{2i\pi}{n}}$相似的性质。
具体的P和G可参考1.11
\begin{lstlisting}
const int mod = 119 << 23 | 1;
const int G = 3;
int wn[20];
void getwn()
{ //  千万不要忘记
    for (int i = 0; i < 20; i++) wn[i] = Pow(G, (mod - 1) / (1 << i), mod);
}
void change(int y[], int len)
{
    for (int i = 1, j = len / 2; i < len - 1; i++)
    {
        if (i < j) swap(y[i], y[j]);
        int k = len / 2;
        while (j >= k) j -= k, k /= 2;
        if (j < k) j += k;
    }
}
void ntt(int y[], int len, int on)
{
    change(y, len);
    for (int h = 2, id = 1; h <= len; h <<= 1, id++)
    {
        for (int j = 0; j < len; j += h)
        {
            int w = 1;
            for (int k = j; k < j + h / 2; k++)
            {
                int u = y[k] % mod;
                int t = 1LL * w * (y[k + h / 2] % mod) % mod;
                y[k] = (u + t) % mod, y[k + h / 2] = ((u - t) % mod + mod) % mod;
                w = 1LL * w * wn[id] % mod;
            }
        }
    }
    if (on == -1)
    {
        //  原本的除法要用逆元
        int inv = Pow(len, mod - 2, mod);
        for (int i = 1; i < len / 2; i++) swap(y[i], y[len - i]);
        for (int i = 0; i < len; i++) y[i] = 1LL * y[i] * inv % mod;
    }
}
\end{lstlisting}
\subsubsection{FWT}
\begin{lstlisting}
void fwt(int f[], int m)
{
    int n = __builtin_ctz(m);
    for (int i = 0; i < n; ++i)
        for (int j = 0; j < m; ++j)
            if (j & (1 << i))
            {
                int l = f[j ^ (1 << i)], r = f[j];
                f[j ^ (1 << i)] = l + r, f[j] = l - r;
                // or: f[j] += f[j ^ (1 << i)];
                // and: f[j ^ (1 << i)] += f[j];
            }
}
void ifwt(int f[], int m)
{
    int n = __builtin_ctz(m);
    for (int i = 0; i < n; ++i)
        for (int j = 0; j < m; ++j)
            if (j & (1 << i))
            {
                int l = f[j ^ (1 << i)], r = f[j];
                f[j ^ (1 << i)] = (l + r) / 2, f[j] = (l - r) / 2;
                // 如果有取模需要使用逆元
                // or: f[j] -= f[j ^ (1 << i)];
                // and: f[j ^ (1 << i)] -= f[j];
            }
}
\end{lstlisting}
\subsection{Numerical Integration}
\subsubsection{Adaptive Simpson's Rule}
\begin{lstlisting}
\end{lstlisting}
$\int _a^b f(x)dx  \approx \frac{b-a}{6}[f(a)+4f(\frac{a+b}{2})+f(b)]$\\
$|S(a, c) + S(c, b) - S(a, b)| / 15 < \epsilon$
\begin{lstlisting}
double F(double x) {}
double simpson(double a, double b)
{ // 三点Simpson法
    double c = a + (b - a) / 2;
    return (F(a) + 4 * F(c) + F(b)) * (b - a) / 6;
}
double asr(double a, double b, double eps, double A)
{ //自适应Simpson公式(递归过程)。已知整个区间[a,b]上的三点Simpson值A
    double c = a + (b - a) / 2;
    double L = simpson(a, c), R = simpson(c, b);
    if (fabs(L + R - A) <= 15 * eps) return L + R + (L + R - A) / 15.0;
    return asr(a, c, eps / 2, L) + asr(c, b, eps / 2, R);
}
double asr(double a, double b, double eps) { return asr(a, b, eps, simpson(a, b)); }
\end{lstlisting}
\subsubsection{Berlekamp-Massey}
\begin{lstlisting}
const int maxn = 1 << 14;
ll res[maxn], base[maxn], _c[maxn], _md[maxn];
vector<int> Md;
void mul(ll* a, ll* b, int k)
{
    for (int i = 0; i < k + k; i++) _c[i] = 0;
    for (int i = 0; i < k; i++)
        if (a[i])
            for (int j = 0; j < k; j++) _c[i + j] = (_c[i + j] + a[i] * b[j]) % mod;
    for (int i = k + k - 1; i >= k; i--)
        if (_c[i])
            for (int j = 0; j < Md.size(); j++) _c[i - k + Md[j]] = (_c[i - k + Md[j]] - _c[i] * _md[Md[j]]) % mod;
    for (int i = 0; i < k; i++) a[i] = _c[i];
}
int solve(ll n, VI a, VI b)
{
    ll ans = 0, pnt = 0;
    int k = a.size();
    assert(a.size() == b.size());
    for (int i = 0; i < k; i++) _md[k - 1 - i] = -a[i];
    _md[k] = 1;
    Md.clear();
    for (int i = 0; i < k; i++)
        if (_md[i] != 0) Md.push_back(i);
    for (int i = 0; i < k; i++) res[i] = base[i] = 0;
    res[0] = 1;
    while ((1LL << pnt) <= n) pnt++;
    for (int p = pnt; p >= 0; p--)
    {
        mul(res, res, k);
        if ((n >> p) & 1)
        {
            for (int i = k - 1; i >= 0; i--) res[i + 1] = res[i];
            res[0] = 0;
            for (int j = 0; j < Md.size(); j++) res[Md[j]] = (res[Md[j]] - res[k] * _md[Md[j]]) % mod;
        }
    }
    for (int i = 0; i < k; i++) ans = (ans + res[i] * b[i]) % mod;
    if (ans < 0) ans += mod;
    return ans;
}
VI BM(VI s)
{
    VI C(1, 1), B(1, 1);
    int L = 0, m = 1, b = 1;
    for (int n = 0; n < s.size(); n++)
    {
        ll d = 0;
        for (int i = 0; i <= L; i++) d = (d + (ll)C[i] * s[n - i]) % mod;
        if (d == 0)
            ++m;
        else if (2 * L <= n)
        {
            VI T = C;
            ll c = mod - d * Pow(b, mod - 2) % mod;
            while (C.size() < B.size() + m) C.push_back(0);
            for (int i = 0; i < B.size(); i++) C[i + m] = (C[i + m] + c * B[i]) % mod;
            L = n + 1 - L, B = T, b = d, m = 1;
        }
        else
        {
            ll c = mod - d * Pow(b, mod - 2) % mod;
            while (C.size() < B.size() + m) C.push_back(0);
            for (int i = 0; i < B.size(); i++) C[i + m] = (C[i + m] + c * B[i]) % mod;
            ++m;
        }
    }
    return C;
}
int gao(VI a, ll n)
{
    VI c = BM(a);
    c.erase(c.begin());
    for (int i = 0; i < c.size(); i++) c[i] = (mod - c[i]) % mod;
    return solve(n, c, VI(a.begin(), a.begin() + c.size()));
}\end{lstlisting}
\subsubsection{Simplex}
\begin{lstlisting}
\end{lstlisting}
输入矩阵$a$描述线性规划的标准形式。\\\\
$a$为$m+1$行$n+1$列,其中行$0 \sim m-1$为不等式,行$m$为目标函数(最大化)。\\\\
列$0 \sim n-1$为变量$0 \sim n-1$的系数,列$n$为常数项。\\\\
约束为$a_{i, 0}x_0 + a_{i, 1}x_1 + \cdots \le a_{i, n}$,目标为$\max(a_{m, 0}x_0 + a_{m, 1}x_1 + \cdots + a_{m, n - 1}x_{n - 1} - a_{m, n})$\\\\
注意:变量均有非负约束$x[i] \ge 0$
\begin{lstlisting}
const int maxm = 500; // 约束数目上限
const int maxn = 500; // 变量数目上限
const double INF = 1e100;
const double eps = 1e-10;
struct Simplex
{
    int n;                // 变量个数
    int m;                // 约束个数
    double a[maxm][maxn]; // 输入矩阵
    int B[maxm], N[maxn]; // 算法辅助变量
    void pivot(int r, int c)
    {
        swap(N[c], B[r]);
        a[r][c] = 1 / a[r][c];
        for (int j = 0; j <= n; j++)
            if (j != c) a[r][j] *= a[r][c];
        for (int i = 0; i <= m; i++)
            if (i != r)
            {
                for (int j = 0; j <= n; j++)
                    if (j != c) a[i][j] -= a[i][c] * a[r][j];
                a[i][c] = -a[i][c] * a[r][c];
            }
    }
    bool feasible()
    {
        for (;;)
        {
            int r, c;
            double p = INF;
            for (int i = 0; i < m; i++)
                if (a[i][n] < p) p = a[r = i][n];
            if (p > -eps) return true;
            p = 0;
            for (int i = 0; i < n; i++)
                if (a[r][i] < p) p = a[r][c = i];
            if (p > -eps) return false;
            p = a[r][n] / a[r][c];
            for (int i = r + 1; i < m; i++)
                if (a[i][c] > eps)
                {
                    double v = a[i][n] / a[i][c];
                    if (v < p) r = i, p = v;
                }
            pivot(r, c);
        }
    }
    // 解有界返回1,无解返回0,无界返回-1。b[i]为x[i]的值,ret为目标函数的值
    int simplex(int n, int m, double x[maxn], double& ret)
    {
        this->n = n, this->m = m;
        for (int i = 0; i < n; i++) N[i] = i;
        for (int i = 0; i < m; i++) B[i] = n + i;
        if (!feasible()) return 0;
        for (;;)
        {
            int r, c;
            double p = 0;
            for (int i = 0; i < n; i++)
                if (a[m][i] > p) p = a[m][c = i];
            if (p < eps)
            {
                for (int i = 0; i < n; i++)
                    if (N[i] < n) x[N[i]] = 0;
                for (int i = 0; i < m; i++)
                    if (B[i] < n) x[B[i]] = a[i][n];
                ret = -a[m][n];
                return 1;
            }
            p = INF;
            for (int i = 0; i < m; i++)
                if (a[i][c] > eps)
                {
                    double v = a[i][n] / a[i][c];
                    if (v < p) r = i, p = v;
                }
            if (p == INF) return -1;
            pivot(r, c);
        }
    }
};\end{lstlisting}
\subsection{Others}
\begin{lstlisting}
\end{lstlisting}
约瑟夫问题\\
$n$个人围成一圈,从第一个开始报数,第$m$个将被杀掉
\begin{lstlisting}
int josephus(int n, int m)
{
    int r = 0;
    for (int k = 1; k <= n; ++k) r = (r + m) % k;
    return r + 1;
}
\end{lstlisting}
$n^n$最左边一位数
\begin{lstlisting}
int leftmost(int n)
{
    double m = n * log10((double)n);
    double g = m - (ll)m;
    return (int)pow(10.0, g);
}
\end{lstlisting}
$n!$位数
\begin{lstlisting}
int count(ll n)
{
    if (n == 1) return 1;
    return (int)ceil(0.5 * log10(2 * M_PI * n) + n * log10(n) - n * log10(M_E));
}
\end{lstlisting}
\subsection{Formula}
\begin{enumerate}
\item 约数定理:若$n=\prod_{i=1}^kp_i^{a_i}$,则

\begin{enumerate}
\item 约数个数$f(n)=\prod_{i=1}^k(a_i+1)$
\item 约数和$g(n)=\prod_{i=1}^k(\sum_{j=0}^{a_i}p_i^j)$
\end{enumerate}

\item 小于$n$且互素的数之和为$n\varphi(n)/2$

\item 若$\gcd(n,i)=1$,则$\gcd(n,n-i)=1(1\leq i\leq n)$

\item 错排公式:$D(n)=(n-1)(D(n-2)+D(n-1))=\sum_{i=2}^n\frac{(-1)^kn!}{k!}=[\frac{n!}{e}+0.5]$

\item 威尔逊定理:$p\ is\ prime\ \Rightarrow (p-1)!\equiv-1\pmod p$

\item 欧拉定理:$\gcd(a,n)=1\Rightarrow a^{\varphi(n)}\equiv1\pmod n$

\item 欧拉定理推广:$\gcd(n,p)=1\Rightarrow a^n\equiv a^{n\%\varphi(p)}\pmod p$

\item 模的幂公式:$a ^ n \pmod {m} = 
\begin{cases}
a ^ n \mod m & n < \varphi(m)\\
a ^ {n \% \varphi(m) + \varphi(m)} \mod m & n \ge \varphi(m)
\end{cases}
$

\item 素数定理:对于不大于$n$的素数个数$\pi(n)$,$\lim\limits_{n\to\infty}\pi(n)=\frac{n}{\ln n}$

\item 位数公式:正整数$x$的位数$N=\log_{10}(n)+1$

\item 斯特灵公式$n!\approx\sqrt{2\pi n}(\frac{n}{e})^n$

\item 设$a>1,m,n>0$,则$\gcd(a^m-1,a^n-1)=a^{\gcd(m,n)}-1$

\item 设$a>b,\gcd(a,b)=1$,则$\gcd(a^m-b^m,a^n-b^n)=a^{\gcd(m,n)}-b^{\gcd(m,n)}$

$$
G=\gcd(C_n^1,C_n^2,...,C_n^{n-1})=
\begin{cases}
	n, & \text{$n$ is prime} \\
	1, & \text{$n$ has multy prime factors} \\
	p, & \text{$n$ has single prime factor $p$}
\end{cases}
$$

$\gcd(Fib(m),Fib(n))=Fib(\gcd(m,n))$

\item 若$\gcd(m,n)=1$,则:

\begin{enumerate}
\item 最大不能组合的数为$m*n-m-n$
\item 不能组合数个数$N=\frac{(m-1)(n-1)}{2}$
\end{enumerate}

\item $(n+1)lcm(C_n^0,C_n^1,...,C_n^{n-1},C_n^{n})=lcm(1,2,...,n+1)$

\item 若$p$为素数,则$(x+y+...+w)^p\equiv x^p+y^p+...+w^p\pmod p$

\item 卡特兰数:1, 1, 2, 5, 14, 42, 132, 429, 1430, 4862, 16796, 58786, 208012

$h(0)=h(1)=1,h(n)=\frac{(4n-2)h(n-1)}{n+1}=\frac{C_{2n}^n}{n+1}=C_{2n}^n-C_{2n}^{n-1}$

\item 伯努利数:$B_n = -\frac{1}{n+1} \sum_{i=0}^{n-1} C_{n+1}^i B_i$

$$\sum_{i=1}^n i^k = \frac{1}{k+1} \sum_{i=1}^{k+1}C_{k+1}^i B_{k+1-i}(n+1)^i$$

\item 二项式反演:$$f_n = \sum_{i = 0} ^ n (-1) ^ i \binom{n}{i} g_i \Leftrightarrow g_n = \sum_{i = 0} ^ n (-1) ^ i \binom{n}{i} f_i$$
$$f_n = \sum_{i = 0} ^ n \binom{n}{i} g_i \Leftrightarrow g_n = \sum_{i = 0} ^ n (-1) ^ {n - i} \binom{n}{i} f_i$$


\item FFT常用素数

\begin{tabular}{cccc}
    \hline
    $r⋅2^k+1$&$r$&$k$&$g$\\
    \hline
    3&1&1&2\\
    5&1&2&2\\
    17&1&4&3\\
    97&3&5&5\\
    193&3&6&5\\
    257&1&8&3\\
    7681&15&9&17\\
    12289&3&12&11\\
    40961&5&13&3\\
    65537&1&16&3\\
    786433&3&18&10\\
    5767169&11&19&3\\
    7340033&7&20&3\\
    23068673&11&21&3\\
    104857601&25&22&3\\
    167772161&5&25&3\\
    469762049&7&26&3\\
    998244353&119&23&3\\
    1004535809&479&21&3\\
    2013265921&15&27&31\\
    2281701377&17&27&3\\
    3221225473&3&30&5\\
    75161927681&35&31&3\\
    77309411329&9&33&7\\
    206158430209&3&36&22\\
    2061584302081&15&37&7\\
    2748779069441&5&39&3\\
    6597069766657&3&41&5\\
    39582418599937&9&42&5\\
    79164837199873&9&43&5\\
    263882790666241&15&44&7\\
    1231453023109121&35&45&3\\
    1337006139375617&19&46&3\\
    3799912185593857&27&47&5\\
    4222124650659841&15&48&19\\
    7881299347898369&7&50&6\\
    31525197391593473&7&52&3\\
    180143985094819841&5&55&6\\
    1945555039024054273&27&56&5\\
    4179340454199820289&29&57&3\\
    \hline
\end{tabular}

\end{enumerate}
\clearpage\section{String Processing}
\subsection{KMP}
\begin{lstlisting}
// 返回y中x的个数
const int N = "Edit";
int next[N];
void initkmp(char x[], int m)
{
    int i = 0, j = next[0] = -1;
    while (i < m)
    {
        while (j != -1 && x[i] != x[j]) j = next[j];
        next[++i] = ++j;
    }
}
int kmp(char x[], int m, char y[], int n)
{
    int i, j, ans;
    i = j = ans = 0;
    initkmp(x, m);
    while (i < n)
    {
        while (j != -1 && y[i] != x[j]) j = next[j];
        i++, j++;
        if (j >= m) ans++, j = next[j];
    }
    return ans;
}
\end{lstlisting}
\subsection{ExtendKMP}
\begin{lstlisting}
//next[i]:x[i...m-1]与x[0...m-1]的最长公共前缀
//extend[i]:y[i...n-1]与x[0...m-1]的最长公共前缀
const int N = "Edit";
int next[N], extend[N];
void pre_ekmp(char x[], int m)
{
    next[0] = m;
    int j = 0;
    while (j + 1 < m && x[j] == x[j + 1]) j++;
    next[1] = j;
    int k = 1;
    for (int i = 2; i < m; i++)
    {
        int p = next[k] + k - 1;
        int L = next[i - k];
        if (i + L < p + 1)
            next[i] = L;
        else
        {
            j = max(0, p - i + 1);
            while (i + j < m && x[i + j] == x[j]) j++;
            next[i] = j;
            k = i;
        }
    }
}
void ekmp(char x[], int m, char y[], int n)
{
    pre_ekmp(x, m, next);
    int j = 0;
    while (j < n && j < m && x[j] == y[j]) j++;
    extend[0] = j;
    int k = 0;
    for (int i = 1; i < n; i++)
    {
        int p = extend[k] + k - 1;
        int L = next[i - k];
        if (i + L < p + 1)
            extend[i] = L;
        else
        {
            j = max(0, p - i + 1);
            while (i + j < n && j < m && y[i + j] == x[j]) j++;
            extend[i] = j, k = i;
        }
    }
}
\end{lstlisting}
\subsection{Manacher}
\begin{lstlisting}
\end{lstlisting}
$O(n)$求解最长回文子串
\begin{lstlisting}
const int N = "Edit";
char s[N], str[N << 1];
int p[N << 1];
void Manacher(char s[], int& n)
{
    str[0] = '$', str[1] = '#';
    for (int i = 0; i < n; i++) str[(i << 1) + 2] = s[i], str[(i << 1) + 3] = '#';
    n = 2 * n + 2;
    str[n] = 0;
    int mx = 0, id;
    for (int i = 1; i < n; i++)
    {
        p[i] = mx > i ? min(p[2 * id - i], mx - i) : 1;
        while (str[i - p[i]] == str[i + p[i]]) p[i]++;
        if (p[i] + i > mx) mx = p[i] + i, id = i;
    }
}
int solve(char s[])
{
    int n = strlen(s);
    Manacher(s, n);
    return *max_elememt(p, p + n) - 1;
}
\end{lstlisting}
\subsection{Aho-Corasick Automaton}
\begin{lstlisting}
const int maxn = "Edit";
struct Trie
{
    int ch[maxn][26], f[maxn], val[maxn];
    int sz, rt;
    int newnode() { memset(ch[sz], -1, sizeof(ch[sz])), val[sz] = 0; return sz++; }
    void init() { sz = 0, rt = newnode(); }
    inline int idx(char c) { return c - 'A'; };
    void insert(const char* s)
    {
        int u = 0;
        for (int i = 0; s[i]; i++)
        {
            int c = idx(s[i]);
            if (ch[u][c] == -1) ch[u][c] = newnode();
            u = ch[u][c];
        }
        val[u]++;
    }
    void build()
    {
        queue<int> q;
        f[rt] = rt;
        for (int c = 0; c < 26; c++)
        {
            if (~ch[rt][c])
                f[ch[rt][c]] = rt, q.push(ch[rt][c]);
            else
                ch[rt][c] = rt;
        }
        while (!q.empty())
        {
            int u = q.front();
            q.pop();
            // val[u] |= val[f[u]];
            for (int c = 0; c < 26; c++)
            {
                if (~ch[u][c])
                    f[ch[u][c]] = ch[f[u]][c], q.push(ch[u][c]);
                else
                    ch[u][c] = ch[f[u]][c];
            }
        }
    }
    //返回主串中有多少模式串
    int query(const char* s)
    {
        int u = rt;
        int res = 0;
        for (int i = 0; s[i]; i++)
        {
            int c = idx(s[i]);
            u = ch[u][c];
            int tmp = u;
            while (tmp != rt)
            {
                res += val[tmp];
                val[tmp] = 0;
                tmp = f[tmp];
            }
        }
        return res;
    }
};
\end{lstlisting}
\subsection{Suffix Array}
\begin{lstlisting}
//倍增算法构造后缀数组,复杂度O(nlogn)
const int maxn = "Edit";
struct Suffix_Array
{
    char s[maxn];
    int sa[maxn], t[maxn], t2[maxn], c[maxn], rank[maxn], height[maxn];
    void build_sa(int m, int n)
    { //n为字符串的长度,字符集的值为0~m-1
        n++;
        int *x = t, *y = t2;
        //基数排序
        for (int i = 0; i < m; i++) c[i] = 0;
        for (int i = 0; i < n; i++) c[x[i] = s[i]]++;
        for (int i = 1; i < m; i++) c[i] += c[i - 1];
        for (int i = n - 1; ~i; i--) sa[--c[x[i]]] = i;
        for (int k = 1; k <= n; k <<= 1)
        { //直接利用sa数组排序第二关键字
            int p = 0;
            for (int i = n - k; i < n; i++) y[p++] = i;
            for (int i = 0; i < n; i++)
                if (sa[i] >= k) y[p++] = sa[i] - k;
            //基数排序第一关键字
            for (int i = 0; i < m; i++) c[i] = 0;
            for (int i = 0; i < n; i++) c[x[y[i]]]++;
            for (int i = 1; i < m; i++) c[i] += c[i - 1];
            for (int i = n - 1; ~i; i--) sa[--c[x[y[i]]]] = y[i];
            //根据sa和y数组计算新的x数组
            swap(x, y);
            p = 1;
            x[sa[0]] = 0;
            for (int i = 1; i < n; i++)
                x[sa[i]] = y[sa[i - 1]] == y[sa[i]] && y[sa[i - 1] + k] == y[sa[i] + k] ? p - 1 : p++;
            if (p >= n) break; //以后即使继续倍增,sa也不会改变,推出
            m = p;             //下次基数排序的最大值
        }
        n--;
        int k = 0;
        for (int i = 0; i <= n; i++) rank[sa[i]] = i;
        for (int i = 0; i < n; i++)
        {
            if (k) k--;
            int j = sa[rank[i] - 1];
            while (s[i + k] == s[j + k]) k++;
            height[rank[i]] = k;
        }
    }
    
    int dp[maxn][30];
    void initrmq(int n)
    {
        for (int i = 1; i <= n; i++)
            dp[i][0] = height[i];
        for (int j = 1; (1 << j) <= n; j++)
            for (int i = 1; i + (1 << j) - 1 <= n; i++)
                dp[i][j] = min(dp[i][j - 1], dp[i + (1 << (j - 1))][j - 1]);
    }
    int rmq(int l, int r)
    {
        int k = 31 - __builtin_clz(r - l + 1);
        return min(dp[l][k], dp[r - (1 << k) + 1][k]);
    }
    int lcp(int a, int b)
    { // 求两个后缀的最长公共前缀
        a = rank[a], b = rank[b];
        if (a > b) swap(a, b);
        return rmq(a + 1, b);
    }
};
\end{lstlisting}
\subsection{Suffix Automation}
\begin{lstlisting}
const int maxn = "Edit";
struct SAM
{
    int len[maxn << 1], link[maxn << 1], ch[maxn << 1][26];
    int num[maxn << 1]; //每个结点所代表的字符串的出现次数
    int sz, rt, last;
    int newnode(int x = 0)
    {
        len[sz] = x;
        link[sz] = -1;
        memset(ch[sz], -1, sizeof(ch[sz]));
        return sz++;
    }
    void init() { sz = last = 0, rt = newnode(); }
    void reset() { last = 0; }
    void extend(int c)
    {
        int np = newnode(len[last] + 1);
        int p;
        for (p = last; ~p && ch[p][c] == -1; p = link[p]) ch[p][c] = np;
        if (p == -1)
            link[np] = rt;
        else
        {
            int q = ch[p][c];
            if (len[p] + 1 == len[q])
                link[np] = q;
            else
            {
                int nq = newnode(len[p] + 1);
                memcpy(ch[nq], ch[q], sizeof(ch[q]));
                link[nq] = link[q], link[q] = link[np] = nq;
                for (; ~p && ch[p][c] == q; p = link[p]) ch[p][c] = nq;
            }
        }
        last = np;
    }
    int topcnt[maxn], topsam[maxn << 1];
    void build(const char* s)
    { // 加入串后拓扑排序
        memset(topcnt, 0, sizeof(topcnt));
        for (int i = 0; i < sz; i++) topcnt[len[i]]++;
        for (int i = 0; i < maxn - 1; i++) topcnt[i + 1] += topcnt[i];
        for (int i = 0; i < sz; i++) topsam[--topcnt[len[i]]] = i;
        int u = rt;
        for (int i = 0; s[i]; i++) num[u = ch[u][s[i] - 'a']] = 1;
        for (int i = sz - 1; ~i; i--)
        {
            int u = topsam[i];
            if (~link[u]) num[link[u]] += num[u];
        }
    }
};
\end{lstlisting}
\subsection{Palindromic Tree}
\begin{lstlisting}
const int maxn = "Edit";
struct Palindromic_Tree
{
    int ch[maxn][26], f[maxn], len[maxn], s[maxn];
    int cnt[maxn]; // 结点表示的本质不同的回文串的个数(调用count()后)
    int num[maxn]; // 结点表示的最长回文串的最右端点为回文串结尾的回文串个数
    int last, sz, n;
    int newnode(int x)
    {
        memset(ch[sz], 0, sizeof(ch[sz]));
        cnt[sz] = num[sz] = 0, len[sz] = x;
        return sz++;
    }
    void init()
    {
        sz = 0;
        newnode(0), newnode(-1);
        last = n = 0, s[0] = -1, f[0] = 1;
    }
    int get_fail(int u)
    {
        while (s[n - len[u] - 1] != s[n]) u = f[u];
        return u;
    }
    void add(int c)
    { // c-='a'
        s[++n] = c;
        int u = get_fail(last);
        if (!ch[u][c])
        {
            int np = newnode(len[u] + 2);
            f[np] = ch[get_fail(f[u])][c];
            num[np] = num[f[np]] + 1;
            ch[u][c] = np;
        }
        last = ch[u][c];
        cnt[last]++;
    }
    void count()
    {
        for (int i = sz - 1; ~i; i--) cnt[f[i]] += cnt[i];
    }
};\end{lstlisting}
\subsection{Hash}
\begin{lstlisting}
typedef unsigned long long ull;
const ull Seed_Pool[] = {146527, 19260817};
const ull Mod_Pool[] = {1000000009, 998244353};
struct Hash
{
    ull SEED, MOD;
    vector<ull> p, h;
    Hash() {}
    Hash(const string& s, const int& seed_index, const int& mod_index)
    {
        SEED = Seed_Pool[seed_index];
        MOD = Mod_Pool[mod_index]; 
        int n = s.length();
        p.resize(n + 1), h.resize(n + 1);
        p[0] = 1;
        for (int i = 1; i <= n; i++) p[i] = p[i - 1] * SEED % MOD;
        for (int i = 1; i <= n; i++) h[i] = (h[i - 1] * SEED % MOD + s[i - 1]) % MOD;
    }
    ull get(int l, int r) { return (h[r] - h[l] * p[r - l] % MOD + MOD) % MOD; }
    ull substr(int l, int m) { return get(l, l + m); }
};\end{lstlisting}
\clearpage\section{Data Structure}
\subsection{Binary Indexed Tree}
\begin{lstlisting}
\end{lstlisting}
$O(\log n)$查询和修改数组的前缀和
\begin{lstlisting}
// 注意下标应从1开始
template <class T>
struct BIT
{
    vector<T> bit;
    int n;
    void init(int n)
    {
        this->n = n;
        bit.assign(n + 1, 0);
    }
    void update(int x, T v)
    {
        for (; x <= n; x + = x & -x) bit[x] += v
    }
    void query(int x)
    {
        T ret = 0;
        for (; x; x -= x & -x) ret += bit[x];
        return ret;
    }
    // 做权值树状数组时求第k小
    int kth(int k)
    {
        int ret = 0, cnt = 0;
        for (int i = 20; ~i; i--)
        {
            ret ^= (1 << i);
            if (ret > n || cnt + bit[ret] >= k)
                ret ^= (1 << i);
            else
                cnt += bit[ret];
        }
        return ret + 1;
    }
};\end{lstlisting}
\subsection{Segment Tree}
线段树必须要能够裸写,此处仅留矩形面积周长系列备忘。\subsubsection{Area Combination}
\begin{lstlisting}
// 矩形面积并
map<double, int> Hash;
map<int, double> rHash;
struct line
{
    double l, r, h;
    int val;
    line(double l = 0, double r = 0, double h = 0, int val = 0) : l(l), r(r), h(h), val(val) {}
    bool operator<(const line& A) const { return h < A.h; }
};
struct Node
{
    int cover;
    double len;
};
const int maxn = 1000;
Node seg[maxn << 2];
void build(int rt, int l, int r)
{
    seg[rt].cover = seg[rt].len = 0;
    if (l == r) return;
    int mid = l + r >> 1;
    build(lson, l, mid);
    build(rson, mid + 1, r);
}
void pushup(int rt, int l, int r)
{
    if (seg[rt].cover > 0)
        seg[rt].len = rHash[r + 1] - rHash[l]; // [l,r)
    else if (l == r)
        seg[rt].len = 0;
    else
        seg[rt].len = seg[lson].len + seg[rson].len;
}
void update(int rt, int l, int r, int L, int R, int val)
{
    if (L <= l && R >= r)
    {
        seg[rt].cover += val;
        pushup(rt, l, r);
        return;
    }
    int mid = l + r >> 1;
    if (mid >= L) update(lson, l, mid, L, R, val);
    if (mid + 1 <= R) update(rson, mid + 1, r, L, R, val);
    pushup(rt, l, r);
}
int main()
{
    int n, kase = 0;
    while (~scanf("%d", &n))
    {
        if (!n) break;
        double x1, x2, y1, y2;
        vector<line> a;
        set<double> xval;
        for (int i = 0; i < n; i++)
        {
            scanf("%lf%lf%lf%lf", &x1, &y1, &x2, &y2);
            a.emplace_back(x1, x2, y1, 1);
            a.emplace_back(x1, x2, y2, -1);
            xval.insert(x1);
            xval.insert(x2);
        }
        // 离散化
        Hash.clear(), rHash.clear();
        int cnt = 0;
        for (auto& v : xval)
        {
            Hash[v] = ++cnt;
            rHash[cnt] = v;
        }
        sort(a.begin(), a.end());
        build(1, 1, cnt);
        double ans = 0;
        for (int i = 0; i < a.size() - 1; i++)
        {
            update(1, 1, cnt, Hash[a[i].l], Hash[a[i].r] - 1,
                   a[i].val); //[l,r)
            ans += (a[i + 1].h - a[i].h) * seg[1].len;
        }
        printf("Test case #%d\n", ++kase);
        printf("Total explored area: %.2lf\n\n", ans);
    }
}\end{lstlisting}
\subsubsection{Area Intersection}
\begin{lstlisting}
// 矩形面积交
map<double, int> Hash;
map<int, double> rHash;
struct Lines
{
    double l, r, h;
    int val;
    bool operator<(const Lines& A) const { return h < A.h; }
};
struct Node
{
    int cnt;     //  覆盖次数
    double len1; // 覆盖次数大于0的长度
    double len2; // 覆盖次数大于1的长度
};
Node seg[maxn << 2];
void build(int rt, int l, int r)
{
    seg[rt].cnt = seg[rt].len1 = seg[rt].len2 = 0;
    if (l == r) return;
    int mid = l + r >> 1;
    build(lson, l, mid);
    build(rson, mid + 1, r);
}
inline void pushup(int rt, int l, int r)
{
    if (seg[rt].cnt > 1)
        seg[rt].len1 = seg[rt].len2 = rHash[r + 1] - rHash[l];
    else if (seg[rt].cnt == 1)
    {
        seg[rt].len1 = rHash[r + 1] - rHash[l];
        if (l == r)
            seg[rt].len2 = 0;
        else
            seg[rt].len2 = seg[lson].len1 + seg[rson].len1;
    }
    else
    {
        if (l == r)
            seg[rt].len1 = seg[rt].len2 = 0;
        else
        {
            seg[rt].len1 = seg[lson].len1 + seg[rson].len1;
            seg[rt].len2 = seg[lson].len2 + seg[rson].len2;
        }
    }
}
void update(int rt, int l, int r, int L, int R, int val)
{
    if (L <= l && R >= r)
    {
        seg[rt].cnt += val;
        pushup(rt, l, r);
        return;
    }
    int mid = l + r >> 1;
    if (L <= mid) update(lson, l, mid, L, R, val);
    if (R >= mid + 1) update(rson, mid + 1, r, L, R, val);
    pushup(rt, l, r);
}
int main()
{
    int T;
    scanf("%d", &T);
    while (T--)
    {
        int n;
        scanf("%d", &n);
        double x1, x2, y1, y2;
        vector<Lines> line;
        set<double> X;
        for (int i = 1; i <= n; i++)
        {
            scanf("%lf%lf%lf%lf", &x1, &y1, &x2, &y2);
            line.push_back({x1, x2, y1, 1});
            line.push_back({x1, x2, y2, -1});
            X.insert(x1);
            X.insert(x2);
        }
        sort(line.begin(), line.end());
        int cnt = 0;
        Hash.clear();
        rHash.clear();
        for (auto& v : X) Hash[v] = ++cnt, rHash[cnt] = v;
        build(1, 1, cnt);
        double area = 0;
        for (int i = 0; i < line.size() - 1; i++)
        {
            update(1, 1, cnt, Hash[line[i].l], Hash[line[i].r] - 1, line[i].val);
            area += seg[1].len2 * (line[i + 1].h - line[i].h);
        }
        printf("%.2lf\n", area);
    }
}\end{lstlisting}
\subsubsection{Perimeter Combination}
\begin{lstlisting}
// 矩形周长并
int n, m[2];
int sum[maxn << 2], cnt[maxn << 2], all[2][maxn];
struct Seg
{
    int l, r, h, d;
    Seg() {}
    Seg(int l, int r, int h, int d) : l(l), r(r), h(h), d(d) {}
    bool operator<(const Seg& rhs) const { return h < rhs.h; }
} a[2][maxn];
#define lson l, m, rt << 1
#define rson m + 1, r, rt << 1 | 1
void pushup(int p, int l, int r, int rt)
{
    if (cnt[rt])
        sum[rt] = all[p][r + 1] - all[p][l];
    else if (l == r)
        sum[rt] = 0;
    else
        sum[rt] = sum[rt << 1] + sum[rt << 1 | 1];
}
void update(int p, int L, int R, int v, int l, int r, int rt)
{
    if (L <= l && r <= R)
    {
        cnt[rt] += v;
        pushup(p, l, r, rt);
        return;
    }
    int m = l + r >> 1;
    if (L <= m) update(p, L, R, v, lson);
    if (R > m) update(p, L, R, v, rson);
    pushup(p, l, r, rt);
}
int main()
{
    while (scanf("%d", &n) == 1)
    {
        for (int i = 1; i <= n; ++i)
        {
            int x1, y1, x2, y2;
            scanf("%d%d%d%d", &x1, &y1, &x2, &y2);
            all[0][i] = x1, all[0][i + n] = x2;
            all[1][i] = y1, all[1][i + n] = y2;
            a[0][i] = Seg(x1, x2, y1, 1);
            a[0][i + n] = Seg(x1, x2, y2, -1);
            a[1][i] = Seg(y1, y2, x1, 1);
            a[1][i + n] = Seg(y1, y2, x2, -1);
        }
        n <<= 1;
        sort(all[0] + 1, all[0] + 1 + n);
        m[0] = unique(all[0] + 1, all[0] + 1 + n) - all[0] - 1;
        sort(all[1] + 1, all[1] + 1 + n);
        m[1] = unique(all[1] + 1, all[1] + 1 + n) - all[1] - 1;
        sort(a[0] + 1, a[0] + 1 + n);
        sort(a[1] + 1, a[1] + 1 + n);
        int ans = 0;
        for (int i = 0; i < 2; ++i)
        {
            int t = 0, last = 0;
            memset(cnt, 0, sizeof cnt);
            memset(sum, 0, sizeof sum);
            for (int j = 1; j <= n; ++j)
            {
                int l = lower_bound(all[i] + 1, all[i] + 1 + m[i], a[i][j].l) - all[i];
                int r = lower_bound(all[i] + 1, all[i] + 1 + m[i], a[i][j].r) - all[i];
                if (l < r) update(i, l, r - 1, a[i][j].d, 1, m[i], 1);
                t += abs(sum[1] - last);
                last = sum[1];
            }
            ans += t;
        }
        printf("%d\n", ans);
    }
    return 0;
}\end{lstlisting}
\subsection{Splay Tree}
\subsubsection{Splay}
\begin{lstlisting}
const int INF = 0x7fffffff;
struct Splay {
#define root t[0].ch[1]
    private:
        struct Node {
            int ch[2];
            int fa, val, siz, tot;
        };
        Node t[N];
        int cnt, num;
        ii id(int x) {
            return x == t[t[x].fa].ch[1];
        }
        iv update(int x) {
            t[x].siz = t[t[x].ch[0]].siz + t[t[x].ch[1]].siz + t[x].tot;
        }
        ii New(int v, int fa) {
            t[++cnt].fa = fa;
            t[cnt].val = v;
            t[cnt].siz = t[cnt].tot = 1;
            return cnt;
        }
        iv destroy(int x) {
            t[x].ch[0] = t[x].ch[1] = t[x].fa = t[x].val = t[x].siz = t[x].tot = 0;
        }
        iv connect(int fa, int x, int d) {
            t[x].fa = fa;
            t[fa].ch[d] = x;
        }
        iv rotate(int x) {
            int f = t[x].fa;
            int ff = t[f].fa;
            int fson = id(x);
            int ffson = id(f);
            int son = t[x].ch[fson ^ 1];
            connect(f, son, fson);
            connect(x, f, fson ^ 1);
            connect(ff, x, ffson);
            update(f), update(x);
        }
        iv splay(int x, int to) {
            to = t[to].fa;
            while (t[x].fa != to) {
                int f = t[x].fa;
                if (t[f].fa != to) {
                    if (id(f) == id(x)) {
                        rotate(f);
                    } else {
                        rotate(x);
                    }
                }
                rotate(x);
            }
        }
    public:
        ii find(int v) {
            int x = root;
            while (x && t[x].val != v) {
                int d = v < t[x].val ? 0 : 1;
                x = t[x].ch[d];
            }
            if (x) {
                splay(x, root);
            }
            return x;
        }
        ii build(int v) {
            ++num;
            if (!root) {
                return root = New(v, 0);
            }
            int x = root;
            while (1) {
                ++t[x].siz;
                if (t[x].val == v) {
                    ++t[x].tot;
                    return x;
                }
                int d = v < t[x].val ? 0 : 1;
                if (!t[x].ch[d]) {
                    return t[x].ch[d] = New(v, x);
                }
                x = t[x].ch[d];
            }
            return 0;
        }
        iv insert(int v) {
            splay(build(v), root);
        }
        iv del(int v) {
            int x = find(v);
            if (!x) {
                return;
            }
            --num;
            if (t[x].tot > 1) {
                --t[x].tot;
                --t[x].siz;
                return;
            }
            if (!t[x].ch[0]) {
                root = t[x].ch[1];
                t[root].fa = 0;
            } else {
                int L = t[x].ch[0];
                int R = t[x].ch[1];
                while (t[L].ch[1]) {
                    L = t[L].ch[1];
                }
                splay(L, t[x].ch[0]);
                connect(0, L, 1);
                connect(L, R, 1);
                update(L);
            }
            destroy(x);
        }
        ii get_rk(int v) { // 寻找权值为 v 的点的排名
            int x = root, rk = 0;
            while (x) {
                if (v == t[x].val) {
                    rk += t[t[x].ch[0]].siz + 1;
                    break;
                }
                if (v < t[x].val) {
                    x = t[x].ch[0];
                } else {
                    rk += t[t[x].ch[0]].siz + t[x].tot;
                    x = t[x].ch[1];
                }
            }
            if (x) {
                splay(x, root);
            }
            return rk;
        }
        ii get_val(int rk) { // 寻找排名为 rk 的点的权值
            // if (rk > num) {
                // return INF;
            // }
            int x = root;
            while (1) {
                int s = t[t[x].ch[0]].siz;
                if (rk <= s) {
                    x = t[x].ch[0];
                }
                else if (rk <= s + t[x].tot) {
                    break;
                }
                else {
                    rk -= s + t[x].tot;
                    x = t[x].ch[1];
                }
            }
            splay(x, root);
            return t[x].val;
        }
        ii get_pre(int v) { // 寻找权值为 v 的点的前驱结点
            int x = root, pre = -INF;
            while (x) {
                if (t[x].val < v) {
                    pre = t[x].val;
                    x = t[x].ch[1];
                } else {
                    x = t[x].ch[0];
                }
            }
            return pre;
        }
        ii get_nx(int v) { // 寻找权值为 v 的点的后继结点
            int x = root, nx = INF;
            while (x) {
                if (t[x].val > v) {
                    nx = t[x].val;
                    x = t[x].ch[0];
                } else {
                    x = t[x].ch[1];
                }
            }
            return nx;
        }
#undef root
} S;
\end{lstlisting}
\subsubsection{Splay_Reverse}
\begin{lstlisting}
#define ls(x) t[x].ch[0]
#define rs(x) t[x].ch[1]
#define tls(x) t[ls(x)]
#define trs(x) t[rs(x)]
const int INF = 0x7fffffff;
struct Splay {
    int ch[2];
    int fa, val, siz, tag;
} t[N];
int n, m, l, r, rt, cnt;
int a[N];
ii id(int x) {
    return x == rs(t[x].fa);
}
iv update(int x) {
    t[x].siz = tls(x).siz + trs(x).siz + 1;
}
iv push_down(int x) {
    if (t[x].tag) {
        ls(x) ^= rs(x) ^= ls(x) ^= rs(x);
        tls(x).tag ^= 1;
        trs(x).tag ^= 1;
        t[x].tag = 0;
    }
}
iv connect(int fa, int x, int d) {
    t[x].fa = fa;
    t[fa].ch[d] = x;
}
iv rotate(int x) {
    int f = t[x].fa;
    int ff = t[f].fa;
    push_down(x);
    push_down(f);
    int fson = id(x);
    int ffson = id(f);
    int son = t[x].ch[fson ^ 1];
    connect(f, son, fson);
    connect(x, f, fson ^ 1);
    connect(ff, x, ffson);
    update(f), update(x);
}
iv splay(int x, int to) {
#define f t[x].fa
    while (f != to) {
        if (t[f].fa != to) {
            rotate(id(x) == id(f) ? f : x);
        }
        rotate(x);
    }
    if (!to) {
        rt = x;
    }
#undef f
}
int build(int l, int r, int fa) {
    if (l > r) {
        return 0;
    }
    int mid = (l + r) >> 1;
    int x = ++cnt;
    t[x].fa = fa;
    t[x].siz = 1;
    t[x].val = a[mid];
    ls(x) = build(l, mid - 1, x);
    rs(x) = build(mid + 1, r, x);
    update(x);
    return x;
}
ii find(int rk) {
    int x = rt;
    while (1) {
        push_down(x);
        if (rk == tls(x).siz + 1) {
            return x;
        }
        if (rk <= tls(x).siz) {
            x = ls(x);
        } else {
            rk -= tls(x).siz + 1;
            x = rs(x);
        }
    }
}
iv reverse(int l, int r) { // 翻转区间(l, r)
    l = find(l), r = find(r);
    splay(l, 0), splay(r, l);
    t[ls(rs(rt))].tag ^= 1;
}
void print(int x) {
    push_down(x);
    if (ls(x)) {
        print(ls(x));
    }
    if (t[x].val != -INF && t[x].val != INF) {
        printf("%d ", t[x].val);
    }
    if (rs(x)) {
        print(rs(x));
    }
}
int main() {
    n = rd(), m = rd();
    for (int i = 1; i <= n; ++i) {
        a[i + 1] = i;
    }
    a[1] = -INF, a[n + 2] = INF;
    rt = build(1, n + 2, 0);
    while (m--) {
        l = rd(), r = rd();
        reverse(l, r + 2);
    }
    print(rt);
    puts("");
    return 0;
}
\end{lstlisting}
\subsection{Functional Segment Tree}
\begin{lstlisting}
\end{lstlisting}
静态查询区间第$k$小的值\\
必要时进行离散化
\begin{lstlisting}
const int maxn = "Edit";
int a[maxn], rt[maxn];
int cnt;
int lson[maxn << 5], rson[maxn << 5], sum[maxn << 5];
#define Lson l, m, lson[x], lson[y]
#define Rson m + 1, r, rson[x], rson[y]
void update(int p, int l, int r, int& x, int y)
{
    lson[++cnt] = lson[y], rson[cnt] = rson[y], sum[cnt] = sum[y] + 1, x = cnt;
    if (l == r) return;
    int m = (l + r) >> 1;
    if (p <= m) update(p, Lson);
    else update(p, Rson);
}
int query(int l, int r, int x, int y, int k)
{
    if (l == r) return l;
    int m = (l + r) >> 1;
    int s = sum[lson[y]] - sum[lson[x]];
    if (s >= k) return query(Lson, k);
    else return query(Rson, k - s);
}
\end{lstlisting}
\subsection{Sparse Table}
\begin{lstlisting}
const int maxn = "Edit";
int dp[maxn][20];
int a[maxn];
#define N 100005
int d[N][20];
void init(int n) {
    for (int i = 1; i <= n; ++i) {
        d[i][0] = a[i];
    }
    for (int j = 1; j <= lg[N]; ++j) {
        for (int i = 1; i + (1 << j) - 1 <= n; ++i) {
            d[i][j] = max(d[i][j - 1], d[i + (1 << (j - 1))][j - 1]);
        }
    }
}
// 返回[l,r]最大值
int rmq(int l, int r, int op)
{
    int k = 31 - __builtin_clz(r - l + 1);
    return max(dp[l][k], dp[r - (1 << k) + 1][k]);
}
\end{lstlisting}
二维RMQ
\begin{lstlisting}
void init(int n, int m)
{
    for (int i = 0; (1 << i) <= n; i++)
        for (int j = 0; (1 << j) <= m; j++)
        {
            if (i == 0 && j == 0) continue;
            for (int row = 1; row + (1 << i) - 1 <= n; row++)
                for (int col = 1; col + (1 << j) - 1 <= m; col++)
                    if (i)
                        dp[row][col][i][j] = max(dp[row][col][i - 1][j],
                                            dp[row + (1 << (i - 1))][col][i - 1][j]);
                    else
                        dp[row][col][i][j] = max(dp[row][col][i][j - 1],
                                            dp[row][col + (1 << (j - 1))][i][j - 1]);
        }
}
int rmq(int x1, int y1, int x2, int y2)
{
    int kx = 31 - __builtin_clz(x2 - x1 + 1);
    int ky = 31 - __builtin_clz(y2 - y1 + 1);
    int m1 = dp[x1][y1][kx][ky];
    int m2 = dp[x2 - (1 << kx) + 1][y1][kx][ky];
    int m3 = dp[x1][y2 - (1 << ky) + 1][kx][ky];
    int m4 = dp[x2 - (1 << kx) + 1][y2 - (1 << ky) + 1][kx][ky];
    return max({m1, m2, m3, m4});
}
\end{lstlisting}
\subsection{Heavy-Light Decomposition}
\begin{lstlisting}
#define N "Edit"
struct HLD
{
    int n, dfs_clock;
    int sz[maxn], top[maxn], son[maxn], dep[maxn], fa[maxn], id[maxn];
    vector<int> G[maxn];
    // vector<pair<PII, int>> edges; 维护边权时,将其下放为儿子结点的点权
    void init(int n)
    {
        this->n = n, memset(son, -1, sizeof(son)), dfs_clock = 0;
        for (int i = 0; i <= n; i++) G[i].clear();
    }
    void add_edge(int u, int v) { G[u].push_back(v), G[v].push_back(u); }
    void dfs(int u, int p, int d)
    {
        dep[u] = d, fa[u] = p, sz[u] = 1;
        for (auto& v : G[u])
        {
            if (v == p) continue;
            dfs(v, u, d + 1);
            sz[u] += sz[v];
            if (son[u] == -1 || sz[v] > sz[son[u]]) son[u] = v;
        }
    }
    void link(int u, int t)
    {
        top[u] = t, id[u] = ++dfs_clock;
        if (son[u] == -1) return;
        link(son[u], t);
        for (auto& v : G[u])
            if (v != son[u] && v != fa[u]) link(v, v);
    }
    int query_path(int u, int v)
    { // 数据结构相关操作,一般使用线段树或者树状数组
        int ret = 0;
        while (top[u] != top[v])
        {
            if (dep[top[u]] < dep[top[v]]) swap(u, v);
            ret += query(id[top[u]], id[u]);
            u = fa[top[u]];
        }
        if (dep[u] > dep[v]) swap(u, v);
        ret += query(id[u], id[v]);
        /* 边权
        if (u == v) return ret;
        if (dep[u] > dep[v]) swap(u, v);
        ret += query(id[son[u]], id[v]);
        */
        return ret;
    }
};
\end{lstlisting}
\subsection{Link-Cut Tree}
\begin{lstlisting}
\end{lstlisting}
动态维护一个森林
\begin{lstlisting}
const int maxn = "Edit";
struct LCT
{
    int val[maxn], sum[maxn]; // 基于点权
    int rev[maxn], ch[maxn][2], fa[maxn];
    int stk[maxn];
    inline void init(int n)
    { // 初始化点权
        for (int i = 1; i <= n; i++) scanf("%d", val + i);
        for (int i = 1; i <= n; i++)
            fa[i] = ch[i][0] = ch[i][1] = rev[i] = 0;
    }
    inline bool isroot(int x) { return ch[fa[x]][0] != x && ch[fa[x]][1] != x; }
    inline bool get(int x) { return ch[fa[x]][1] == x; }
    void pushdown(int x)
    {
        if (!rev[x]) return;
        swap(ch[x][0], ch[x][1]);
        if (ch[x][0]) rev[ch[x][0]] ^= 1;
        if (ch[x][1]) rev[ch[x][1]] ^= 1;
        rev[x] ^= 1;
    }
    void pushup(int x) { sum[x] = val[x] + sum[ch[x][0]] + sum[ch[x][1]]; }
    void rotate(int x)
    {
        int y = fa[x], z = fa[fa[x]], d = get(x);
        if (!isroot(y)) ch[z][get(y)] = x;
        fa[x] = z;
        ch[y][d] = ch[x][d ^ 1], fa[ch[y][d]] = y;
        ch[x][d ^ 1] = y, fa[y] = x;
        pushup(y), pushup(x);
    }
    void splay(int x)
    {
        int top = 0;
        stk[++top] = x;
        for (int i = x; !isroot(i); i = fa[i]) stk[++top] = fa[i];
        for (int i = top; i; i--) pushdown(stk[i]);
        for (int f; !isroot(x); rotate(x))
            if (!isroot(f = fa[x])) rotate(get(x) == get(f) ? f : x);
    }
    void access(int x)
    {
        for (int y = 0; x; y = x, x = fa[x]) splay(x), ch[x][1] = y, pushup(x);
    }
    int find(int x)
    {
        access(x), splay(x);
        while (ch[x][0]) x = ch[x][0];
        return x;
    }
    void makeroot(int x) { access(x), splay(x), rev[x] ^= 1; }
    void link(int x, int y) { makeroot(x), fa[x] = y, splay(x); }
    void cut(int x, int y) { makeroot(x), access(y), splay(y), fa[x] = ch[y][0] = 0; }
    void update(int x, int v) { val[x] = v, access(x), splay(x); }
    int query(int x, int y)
    {
        makeroot(y), access(x), splay(x);
        return sum[x];
    }
};
\end{lstlisting}
\subsection{Virtual Tree}
\begin{lstlisting}
const int maxn = "Edit";
vector<int> vtree[maxn];
void build(vector<int>& vec)
{
    sort(vec.begin(), vec.end(), [&](int x, int y) { return dfn[x] < dfn[y]; });
    static int s[maxn];
    int top = 0;
    s[top] = 0;
    vtree[0].clear();
    for (auto& u : vec)
    {
        int vlca = lca(u, s[top]);
        vtree[u].clear();
        if (vlca == s[top])
            s[++top] = u;
        else
        {
            while (top && dep[s[top - 1]] >= dep[vlca])
            {
                vtree[s[top - 1]].push_back(s[top]);
                top--;
            }
            if (s[top] != vlca)
            {
                vtree[vlca].clear();
                vtree[vlca].push_back(s[top--]);
                s[++top] = vlca;
            }
            s[++top] = u;
        }
    }
    for (int i = 0; i < top; ++i) vtree[s[i]].push_back(s[i + 1]);
}
\end{lstlisting}
\subsection{Cartesian Tree}
\begin{lstlisting}
const int maxn = "Edit";
int lson[maxn], rson[maxn], fa[maxn];
void build(int n)
{
    stack<int> s;
    for (int i = 0; i < n; i++)
    {
        int last = -1;
        while (!s.empty() && a[i] > a[s.top()]) last = s.top(), s.pop();
        if (!s.empty()) rson[s.top()] = i, fa[i] = s.top();
        lson[i] = last;
        if (~last) fa[last] = i;
        s.push(i);
    }
}\end{lstlisting}
\clearpage\section{Graph Theory}
\subsection{Shortest Path}
\begin{lstlisting}
struct Edge
{
    int from, to, dist;
    Edge() {}
    Edge(int u, int v, int d) : from(u), to(v), dist(d) {}
};\end{lstlisting}
\subsubsection{Dijkstra}
\begin{lstlisting}
struct HeapNode
{
    int d, u;
    bool operator<(const HeapNode& rhs) const
    {
        return d > rhs.d;
    }
};
const int maxn = "Edit";
struct Dijkstra
{
    int n, m;            // 点数和边数
    vector<Edge> edges;  // 边列表
    vector<int> G[maxn]; // 每个节点出发的边编号(从0开始编号)
    bool done[maxn];     // 是否已永久标号
    int d[maxn];         // s到各点的距离
    int p[maxn];         // 最短路中的一条边
    void init(int n)
    {
        this->n = n;
        for (int i = 0; i < n; i++) G[i].clear(); // 清空邻接表
        edges.clear();                            // 清空边表
    }
    void AddEdge(int from, int to, int dist)
    { // 如果是无向图,每条无向边需调用两次AddEdge
        edges.emplace_back(from, to, dist);
        m = edges.size();
        G[from].push_back(m - 1);
    }
    void dijkstra(int s)
    {
        priority_queue<HeapNode> q;
        for (int i = 0; i < n; i++) d[i] = INF;
        d[s] = 0;
        memset(done, 0, sizeof(done));
        q.push({0, s});
        while (!q.empty())
        {
            HeapNode x = q.top();
            q.pop();
            int u = x.u;
            if (done[u]) continue;
            done[u] = true;
            for (auto& id : G[u])
            {
                Edge& e = edges[id];
                if (d[e.to] > d[u] + e.dist)
                {
                    d[e.to] = d[u] + e.dist;
                    p[e.to] = id;
                    q.push({d[e.to], e.to});
                }
            }
        }
    }
};
\end{lstlisting}
\subsubsection{Bellman-Ford}
\begin{lstlisting}
const int maxn = "Edit";
struct BellmanFord
{
    int n, m;
    vector<Edge> edges;
    vector<int> G[maxn];
    bool inq[maxn]; // 是否在队列中
    int d[maxn];    // s到各个点的距离
    int p[maxn];    // 最短路中的上一条弧
    int cnt[maxn];  // 进队次数
    void init(int n)
    {
        this->n = n;
        for (int i = 0; i < n; i++) G[i].clear();
        edges.clear();
    }
    void AddEdge(int from, int to, int dist)
    {
        edges.emplace_back(from, to, dist);
        m = edges.size();
        G[from].push_back(m - 1);
    }
    bool bellmanford(int s)
    {
        queue<int> q;
        memset(inq, 0, sizeof(inq));
        memset(cnt, 0, sizeof(cnt));
        for (int i = 0; i < n; i++) d[i] = INF;
        d[s] = 0;
        inq[s] = true;
        q.push(s);
        while (!q.empty())
        {
            int u = q.front();
            q.pop();
            inq[u] = false;
            for (auto& id : G[u])
            {
                Edge& e = edges[id];
                if (d[u] < INF && d[e.to] > d[u] + e.dist)
                {
                    d[e.to] = d[u] + e.dist;
                    p[e.to] = id;
                    if (!inq[e.to])
                    {
                        q.push(e.to);
                        inq[e.to] = true;
                        if (++cnt[e.to] > n) return false;
                    }
                }
            }
        }
        return true;
    }
};
\end{lstlisting}
\subsection{Minimal Spanning Tree}
\subsubsection{Zhu Liu}
\begin{lstlisting}
const int maxn = "Edit";
// 固定根的最小树型图,邻接矩阵写法
struct MDST
{
    int n;
    int w[maxn][maxn]; // 边权
    int vis[maxn];     // 访问标记,仅用来判断无解
    int ans;           // 计算答案
    int removed[maxn]; // 每个点是否被删除
    int cid[maxn];     // 所在圈编号
    int pre[maxn];     // 最小入边的起点
    int iw[maxn];      // 最小入边的权值
    int max_cid;       // 最大圈编号
    void init(int n)
    {
        this->n = n;
        for (int i = 0; i < n; i++)
            for (int j = 0; j < n; j++) w[i][j] = INF;
    }
    void AddEdge(int u, int v, int cost)
    {
        w[u][v] = min(w[u][v], cost); // 重边取权最小的
    }
    // 从s出发能到达多少个结点
    int dfs(int s)
    {
        vis[s] = 1;
        int ans = 1;
        for (int i = 0; i < n; i++)
            if (!vis[i] && w[s][i] < INF) ans += dfs(i);
        return ans;
    }
    // 从u出发沿着pre指针找圈
    bool cycle(int u)
    {
        max_cid++;
        int v = u;
        while (cid[v] != max_cid)
        {
            cid[v] = max_cid;
            v = pre[v];
        }
        return v == u;
    }
    // 计算u的最小入弧,入弧起点不得在圈c中
    void update(int u)
    {
        iw[u] = INF;
        for (int i = 0; i < n; i++)
            if (!removed[i] && w[i][u] < iw[u])
            {
                iw[u] = w[i][u];
                pre[u] = i;
            }
    }
    // 根结点为s,如果失败则返回false
    bool solve(int s)
    {
        memset(vis, 0, sizeof(vis));
        if (dfs(s) != n) return false;
        memset(removed, 0, sizeof(removed));
        memset(cid, 0, sizeof(cid));
        for (int u = 0; u < n; u++) update(u);
        pre[s] = s;
        iw[s] = 0; // 根结点特殊处理
        ans = max_cid = 0;
        for (;;)
        {
            bool have_cycle = false;
            for (int u = 0; u < n; u++)
                if (u != s && !removed[u] && cycle(u))
                {
                    have_cycle = true;
                    // 以下代码缩圈,圈上除了u之外的结点均删除
                    int v = u;
                    do
                    {
                        if (v != u) removed[v] = 1;
                        ans += iw[v];
                        // 对于圈外点i,把边i->v改成i->u(并调整权值);v->i改为u->i
                        // 注意圈上可能还有一个v'使得i->v'或者v'->i存在,
                        // 因此只保留权值最小的i->u和u->i
                        for (int i = 0; i < n; i++)
                            if (cid[i] != cid[u] && !removed[i])
                            {
                                if (w[i][v] < INF)
                                    w[i][u] = min(w[i][u], w[i][v] - iw[v]);
                                w[u][i] = min(w[u][i], w[v][i]);
                                if (pre[i] == v) pre[i] = u;
                            }
                        v = pre[v];
                    } while (v != u);
                    update(u);
                    break;
                }
            if (!have_cycle) break;
        }
        for (int i = 0; i < n; i++)
            if (!removed[i]) ans += iw[i];
        return true;
    }
};\end{lstlisting}
\subsection{LCA}
\subsubsection{DFS+ST}
\begin{lstlisting}
\end{lstlisting}
DFS+ST在线算法\\
时间复杂度$O(nlogn+q)$
\begin{lstlisting}
const int maxn = "Edit";
vector<int> G[maxn], sp;
int dep[maxn], dfn[maxn];
PII dp[21][maxn << 1];
void init(int n)
{
    for (int i = 0; i < n; i++) G[i].clear();
    sp.clear();
}
void dfs(int u, int fa)
{
    dep[u] = dep[fa] + 1;
    dfn[u] = sp.size();
    sp.push_back(u);
    for (auto& v : G[u])
    {
        if (v == fa) continue;
        dfs(v, u);
        sp.push_back(u);
    }
}
void initrmq()
{
    int n = sp.size();
    for (int i = 0; i < n; i++) dp[0][i] = {dfn[sp[i]], sp[i]};
    for (int i = 1; (1 << i) <= n; i++)
        for (int j = 0; j + (1 << i) - 1 < n; j++)
            dp[i][j] = min(dp[i - 1][j], dp[i - 1][j + (1 << (i - 1))]);
}
int lca(int u, int v)
{
    int l = dfn[u], r = dfn[v];
    if (l > r) swap(l, r);
    int k = 31 - __builtin_clz(r - l + 1);
    return min(dp[k][l], dp[k][r - (1 << k) + 1]).second;
}
\end{lstlisting}
\subsubsection{Tarjan}
\begin{lstlisting}
\end{lstlisting}
Tarjan离线算法\\
时间复杂度$O(n+q)$
\begin{lstlisting}
const int maxn = "Edit";
int par[maxn];           //并查集
int ans[maxn];           //存储答案
vector<int> G[maxn];     //邻接表
vector<PII> query[maxn]; //存储查询信息
bool vis[maxn];          //是否被遍历
inline void init(int n)
{
    for (int i = 1; i <= n; i++)
    {
        G[i].clear(), query[i].clear();
        par[i] = i, vis[i] = 0;
    }
}
inline void add_edge(int u, int v) { G[u].push_back(v); }
inline void add_query(int id, int u, int v)
{
    query[u].emplace_back(v, id);
    query[v].emplace_back(u, id);
}
void tarjan(int u)
{
    vis[u] = 1;
    for (auto& v : G[u])
    {
        if (vis[v]) continue;
        tarjan(v);
        unite(u, v);
    }
    for (auto& q : query[u])
    {
        int &v = q.X, &id = q.Y;
        if (!vis[v]) continue;
        ans[id] = find(v);
    }
}
\end{lstlisting}
\subsection{Depth-First Traversal}
\subsubsection{Biconnected-Component}
\begin{lstlisting}
//割顶的bccno无意义
const int maxn = "Edit";
int pre[maxn], iscut[maxn], bccno[maxn], dfs_clock, bcc_cnt;
vector<int> G[maxn], bcc[maxn];
stack<PII> s;
void init(int n)
{
    for (int i = 0; i < n; i++) G[i].clear();
}
inline void add_edge(int u, int v) { G[u].push_back(v), G[v].push_back(u); }
int dfs(int u, int fa)
{
    int lowu = pre[u] = ++dfs_clock;
    int child = 0;
    for (auto& v : G[u])
    {
        PII e = {u, v};
        if (!pre[v])
        {
            //没有访问过v
            s.push(e);
            child++;
            int lowv = dfs(v, u);
            lowu = min(lowu, lowv); //用后代的low函数更新自己
            if (lowv >= pre[u])
            {
                iscut[u] = true;
                bcc_cnt++;
                bcc[bcc_cnt].clear(); //注意!bcc从1开始编号
                for (;;)
                {
                    PII x = s.top();
                    s.pop();
                    if (bccno[x.first] != bcc_cnt)
                        bcc[bcc_cnt].push_back(x.first), bcc[x.first] = bcc_cnt;
                    if (bccno[x.second] != bcc_cnt)
                        bcc[bcc_cnt].push_back(x.second), bcc[x.second] = bcc_cnt;
                    if (x.first == u && x.second == v) break;
                }
            }
        }
        else if (pre[v] < pre[u] && v != fa)
        {
            s.push(e);
            lowu = min(lowu, pre[v]); //用反向边更新自己
        }
    }
    if (fa < 0 && child == 1) iscut[u] = 0;
    return lowu;
}
void find_bcc(int n)
{
    //调用结束后s保证为空,所以不用清空
    memset(pre, 0, sizeof(pre));
    memset(iscut, 0, sizeof(iscut));
    memset(bccno, 0, sizeof(bccno));
    dfs_clock = bcc_cnt = 0;
    for (int i = 0; i < n; i++)
        if (!pre[i]) dfs(i, -1);
}
\end{lstlisting}
\subsubsection{Strongly Connected Component}
\begin{lstlisting}
const int maxn = "Edit";
vector<int> G[maxn];
int pre[maxn], lowlink[maxn], sccno[maxn], dfs_clock, scc_cnt;
stack<int> S;
inline void init(int n)
{
    for (int i = 0; i < n; i++) G[i].clear();
}
inline void add_edge(int u, int v) { G[u].push_back(v); }
void dfs(int u)
{
    pre[u] = lowlink[u] = ++dfs_clock;
    S.push(u);
    for (auto& v : G[u])
    {
        if (!pre[v])
        {
            dfs(v);
            lowlink[u] = min(lowlink[u], lowlink[v]);
        }
        else if (!sccno[v])
            lowlink[u] = min(lowlink[u], pre[v]);
    }
    if (lowlink[u] == pre[u])
    {
        scc_cnt++;
        for (;;)
        {
            int x = S.top();
            S.pop();
            sccno[x] = scc_cnt;
            if (x == u) break;
        }
    }
}
void find_scc(int n)
{
    dfs_clock = 0, scc_cnt = 0;
    memset(sccno, 0, sizeof(sccno)), memset(pre, 0, sizeof(pre));
    for (int i = 0; i < n; i++)
        if (!pre[i]) dfs(i);
}
\end{lstlisting}
\subsubsection{2-SAT}
\begin{lstlisting}
const int maxn = "Edit";
struct TwoSAT
{
    int n;
    vector<int> G[maxn << 1];
    bool mark[maxn << 1];
    int S[maxn << 1], c;
    void init(int n)
    {
        this->n = n;
        for (int i = 0; i < (n << 1); i++) G[i].clear();
        memset(mark, 0, sizeof(mark));
    }
    bool dfs(int x)
    {
        if (mark[x ^ 1]) return false;
        if (mark[x]) return true;
        mark[x] = true;
        S[c++] = x;
        for (auto& y : G[x])
            if (!dfs(y)) return false;
        return true;
    }
    //x = xval or y = yval
    void add_clause(int x, int xval, int y, int yval)
    {
        x = (x << 1) + xval;
        y = (y << 1) + yval;
        G[x ^ 1].push_back(y);
        G[y ^ 1].push_back(x);
    }
    bool solve()
    {
        for (int i = 0; i < (n << 1); i += 2)
            if (!mark[i] && !mark[i + 1])
            {
                c = 0;
                if (!dfs(i))
                {
                    while (c > 0) mark[S[--c]] = false;
                    if (!dfs(i + 1)) return false;
                }
            }
        return true;
    }
};
\end{lstlisting}
\subsection{Eular Path}
\begin{itemize}
\item  基本概念:
    \begin{itemize}
    \item 欧拉图: 能够没有重复地一次遍历所有边的图。(必须是连通图)
    \item 欧拉路: 上述遍历的路径就是欧拉路。
    \item 欧拉回路: 若欧拉路是闭合的(一个圈,从起点开始遍历最终又回到起点),则为欧拉回路。
    \end{itemize}
\item 无向图G有欧拉路径的充要条件
    \begin{itemize}
    \item G是连通图
    \item G中奇顶点(连接边的数量为奇数)的数量等于0或2.
    \end{itemize}
\item 无向图G有欧拉回路的充要条件
    \begin{itemize}
    \item G是连通图
    \item G中每个顶点都是偶顶点
    \end{itemize}
\item 有向图G有欧拉路径的充要条件
    \begin{itemize}
    \item G是连通图
    \item u的出度比入度大1,v的出度比入度小1,其他所有点出度和入度相同。(u为起点,v为终点)
    \end{itemize}
\item 有向图G有欧拉回路的充要条件
    \begin{itemize}
    \item G是连通图
    \item G中每个顶点的出度等于入度
    \end{itemize}
\end{itemize}
\subsubsection{Fleury}
\begin{lstlisting}
\end{lstlisting}
若有两个点的度数是奇数,则此时这两个点只能作为欧拉路径的起点和终点。
\begin{lstlisting}
const int maxn = "Edit";
int G[maxn][maxn];
int deg[maxn][maxn];
vector<int> ans;
inline void init() { memset(G, 0, sizeof(G)), memset(deg, 0, sizeof(deg)); }
inline void AddEdge(int u, int v) { deg[u]++, deg[v]++, G[u][v]++, G[v][u]++; }
void Fleury(int s)
{
    for (int i = 0; i < n; i++)
        if (G[s][i])
        {
            G[s][i]--, G[i][s]--;
            Fleury(i);
        }
    ans.push_back(s);
}
\end{lstlisting}
\subsection{Bipartite Graph Matching}
\begin{enumerate}
\item 一个二分图中的最大匹配数等于这个图中的最小点覆盖数

\item 最小路径覆盖=$|G|$-最大匹配数

在一个 $N \times N$ 的有向图中,路径覆盖就是在图中找一些路经,使之覆盖了图中的所有顶点,且任何一个顶点有且只有一条路径与之关联;

(如果把这些路径中的每条路径从它的起始点走到它的终点,那么恰好可以经过图中的每个顶点一次且仅一次);如果不考虑图中存在回路,那么每每条路径就是一个弱连通子集.

由上面可以得出:

\begin{enumerate}
\item 一个单独的顶点是一条路径;

\item 如果存在一路径 $p_1,p_2,......p_k$,其中 $p_1$ 为起点,$p_k$ 为终点,那么在覆盖图中,顶点 $p_1,p2,......p_k$ 不再与其它的顶点之间存在有向边.
\end{enumerate}

最小路径覆盖就是找出最小的路径条数,使之成为 $G$ 的一个路径覆盖.

路径覆盖与二分图匹配的关系:最小路径覆盖=$|G|$-最大匹配数;

\item 二分图最大独立集=顶点数-二分图最大匹配

独立集:图中任意两个顶点都不相连的顶点集合。
\end{enumerate}
\subsubsection{Hungry(Matrix)}
\begin{lstlisting}
\end{lstlisting}
时间复杂度:$O(VE)$.\\
顶点编号从0开始
\begin{lstlisting}
const int maxn = "Edit";
int uN, vN;        //uN是匹配左边的顶点数,vN是匹配右边的顶点数
int g[maxn][maxn]; //邻接矩阵g[i][j]表示i->j的有向边就可以了,是左边向右边的匹配
int linker[maxn];
bool used[maxn];
bool dfs(int u)
{
    for (int v = 0; v < vN; v++)
        if (g[u][v] && !used[v])
        {
            used[v] = true;
            if (linker[v] == -1 || dfs(linker[v]))
            {
                linker[v] = u;
                return true;
            }
        }
    return false;
}
int hungary()
{
    int res = 0;
    memset(linker, -1, sizeof(linker));
    for (int u = 0; u < uN; u++)
    {
        memset(used, 0, sizeof(used));
        if (dfs(u)) res++;
    }
    return res;
}
\end{lstlisting}
\subsubsection{Hungry(List)}
\begin{lstlisting}
\end{lstlisting}
使用前用init()进行初始化\\
加边使用函数addedge(u,v)
\begin{lstlisting}
const int maxn = "Edit";
int n;
vector<int> G[maxn];
int linker[maxn];
bool used[maxn];
inline void init(int n)
{
    for (int i = 0; i < n; i++) G[i].clear();
}
inline void addedge(int u, int v) { G[u].push_back(v); }
bool dfs(int u)
{
    for (auto& v : G[u])
    {
        if (!used[v])
        {
            used[v] = true;
            if (linker[v] == -1 || dfs(linker[v]))
            {
                linker[v] = u;
                return true;
            }
        }
    }
    return false;
}
int hungary()
{
    int ans = 0;
    memset(linker, -1, sizeof(linker));
    for (int u = 0; u < n; u++)
    {
        memset(used, 0, sizeof(used));
        if (dfs(u)) ans++;
    }
    return ans;
}
\end{lstlisting}
\subsubsection{Hopcroft-Carp}
\begin{lstlisting}
\end{lstlisting}
复杂度$O(\sqrt{n}*E)$\\
$uN$ 为左端的顶点数,使用前赋值(点编号0开始)
\begin{lstlisting}
const int maxn = "Edit";
vector<int> G[maxn];
int uN, dis;
int Mx[maxn], My[maxn];
int dx[maxn], dy[maxn];
bool used[maxn];
inline void init(int n)
{
    for (int i = 0; i < n; i++) G[i].clear();
}
inline void addedge(int u, int v) { G[u].push_back(v); }
bool bfs()
{
    queue<int> q;
    dis = INF;
    memset(dx, -1, sizeof(dx)), memset(dy, -1, sizeof(dy));
    for (int i = 0; i < uN; i++)
        if (Mx[i] == -1) q.push(i), dx[i] = 0;
    while (!q.empty())
    {
        int u = q.front();
        q.pop();
        if (dx[u] > dis) break;
        for (auto& v : G[u])
        {
            if (dy[v] == -1)
            {
                dy[v] = dx[u] + 1;
                if (My[v] == -1)
                    dis = dy[v];
                else
                {
                    dx[My[v]] = dy[v] + 1;
                    q.push(My[v]);
                }
            }
        }
    }
    return dis != INF;
}
bool dfs(int u)
{
    for (auto& v : G[u])
    {
        if (!used[v] && dy[v] == dx[u] + 1)
        {
            used[v] = true;
            if (My[v] != -1 && dy[v] == dis) continue;
            if (My[v] == -1 || dfs(My[v]))
            {
                My[v] = u, Mx[u] = v;
                return true;
            }
        }
    }
    return false;
}
int MaxMatch()
{
    int res = 0;
    memset(Mx, -1, sizeof(Mx)), memset(My, -1, sizeof(My));
    while (bfs())
    {
        memset(used, false, sizeof(used));
        for (int i = 0; i < uN; i++)
            if (Mx[i] == -1 && dfs(i)) res++;
    }
    return res;
}
\end{lstlisting}
\subsubsection{Hungry(Multiple)}
\begin{lstlisting}
const int maxn = "Edit";
const int maxm = "Edit";
int uN, vN;        //u,v的数目,使用前面必须赋值
int g[maxn][maxm]; //邻接矩阵
int linker[maxm][maxn];
bool used[maxm];
int num[maxm]; //右边最大的匹配数
bool dfs(int u)
{
    for (int v = 0; v < vN; v++)
        if (g[u][v] && !used[v])
        {
            used[v] = true;
            if (linker[v][0] < num[v])
            {
                linker[v][++linker[v][0]] = u;
                return true;
            }
            for (int i = 1; i <= num[0]; i++)
                if (dfs(linker[v][i]))
                {
                    linker[v][i] = u;
                    return true;
                }
        }
    return false;
}
int hungary()
{
    int res = 0;
    for (int i = 0; i < vN; i++) linker[i][0] = 0;
    for (int u = 0; u < uN; u++)
    {
        memset(used, 0, sizeof(used));
        if (dfs(u)) res++;
    }
    return res;
}
\end{lstlisting}
\subsubsection{Kuhn-Munkres}
\begin{lstlisting}
const int maxn = "Edit";
int n;
int cost[maxn][maxn];
int lx[maxn], ly[maxn], match[maxn], slack[maxn];
int prev[maxn];
bool vy[maxn];

void augment(int root)
{
    fill(vy + 1, vy + n + 1, false);
    fill(slack + 1, slack + n + 1, INF);
    int py;
    match[py = 0] = root;
    do
    {
        vy[py] = true;
        int x = match[py], yy;
        int delta = INF;
        for (int y = 1; y <= n; y++)
        {
            if (!vy[y])
            {
                if (lx[x] + ly[y] - cost[x][y] < slack[y])
                    slack[y] = lx[x] + ly[y] - cost[x][y], prev[y] = py;
                if (slack[y] < delta) delta = slack[y], yy = y;
            }
        }
        for (int y = 0; y <= n; y++)
        {
            if (vy[y])
                lx[match[y]] -= delta, ly[y] += delta;
            else
                slack[y] -= delta;
        }
        py = yy;
    } while (match[py] != -1);
    do
    {
        int pre = prev[py];
        match[py] = match[pre], py = pre;
    } while (py);
}
int KM()
{
    for (int i = 1; i <= n; i++)
    {
        lx[i] = ly[i] = 0;
        match[i] = -1;
        for (int j = 1; j <= n; j++) lx[i] = max(lx[i], cost[i][j]);
    }
    int answer = 0;
    for (int root = 1; root <= n; root++) augment(root);
    for (int i = 1; i <= n; i++) answer += lx[i], answer += ly[i];
    return answer;
}\end{lstlisting}
\subsection{Network Flow}
\begin{lstlisting}
struct Edge
{
    int from, to, cap, flow;
    Edge(int u, int v, int c, int f)
        : from(u), to(v), cap(c), flow(f) {}
};
\end{lstlisting}
费用流
\begin{lstlisting}
struct Edge
{
    int from, to, cap, flow, cost;
    Edge(int u, int v, int c, int f, int w)
        : from(u), to(v), cap(c), flow(f), cost(w) {}
};
\end{lstlisting}
\subsubsection*{建模技巧}


\indent

\textbf{二分图带权最大独立集}。给出一个二分图,每个结点上有一个正权值。要求选出一些点,使得这些点之间没有边相连,且权值和最大。

\indent

\textbf{解:}在二分图的基础上添加源点$S$和汇点$T$,然后从$S$向所有$X$集合中的点连一条边,所有$Y$集合中的点向$T$连一条边,容量均为该点的权值。$X$结点与$Y$结点之间的边的容量均为无穷大。这样,对于图中的任意一个割,将割中的边对应的结点删掉就是一个符合要求的解,权和为所有权减去割的容量。因此,只需要求出最小割,就能求出最大权和。

\indent

\textbf{公平分配问题}。把$m$个任务分配给$n$个处理器。其中每个任务有两个候选处理器,可以任选一个分配。要求所有处理器中,任务数最多的那个处理器所分配的任务数尽量少。不同任务的候选处理器集$\lbrace p_1 , p_2 \rbrace$保证不同。

\indent

\textbf{解:}本题有一个比较明显的二分图模型,即$X$结点是任务,$Y$结点是处理器。二分答案$x$,然后构图,首先从源点$S$出发向所有的任务结点引一条边,容量等于$1$,然后从每个任务结点出发引两条边,分别到达它所能分配到的两个处理器结点,容量为$1$,最后从每个处理器结点出发引一条边到汇点$T$,容量为$x$,表示选择该处理器的任务不能超过$x$。这样网络中的每个单位流量都是从$S$流到一个任务结点,再到处理器结点,最后到汇点$T$。只有当网络中的总流量等于$m$时才意味着所有任务都选择了一个处理器。这样,我们通过$O(\log m)$次最大流便算出了答案。

\indent

\textbf{区间$k$覆盖问题}。数轴上有一些带权值的左闭右开区间。选出权和尽量大的一些区间,使得任意一个数最多被k个区间覆盖。

\indent

\textbf{解:}本题可以用最小费用流解决,构图方法是把每个数作为一个结点,然后对于权值为$w$的区间$[u,v)$加边$u→v$,容量为$1$,费用为$-w$。再对所有相邻的点加边$i→i+1$,容量为$k$,费用为$0$。最后,求最左点到最右点的最小费用最大流即可,其中每个流量对应一组互不相交的区间。如果数值范围太大,可以先进行离散化。

\indent

\textbf{最大闭合子图}。给定带权图$G$(权值可正可负),求一个权和最大的点集,使得起点在该点集中的任意弧,终点也在该点集中。

\indent

\textbf{解:}新增附加源$s$和附加汇$t$,从$s$向所有正权点引一条边,容量为权值;从所有负权点向汇点引一条边,容量为权值的相反数。求出最小割以后,$S - \lbrace s \rbrace$就是最大闭合子图。

\indent

\textbf{最大密度子图}。给出一个无向图,找一个点集,使得这些点之间的边数除以点数的值(称为子图的密度)最大。

\indent

\textbf{解:}如果两个端点都选了,就必然要选边,这就是一种推导。如果把每个点和每条边都看成新图中的结点,可以把问题转化为最大闭合子图。\subsubsection{EdmondKarp}
\begin{lstlisting}
const int maxn = "Edit";
struct EdmonsKarp //时间复杂度O(v*E*E)
{
    int n, m;
    vector<Edge> edges;  //边数的两倍
    vector<int> G[maxn]; //邻接表,G[i][j]表示节点i的第j条边在e数组中的序号
    int a[maxn];         //起点到i的可改进量
    int p[maxn];         //最短路树上p的入弧编号
    void init(int n)
    {
        for (int i = 0; i < n; i++) G[i].clear();
        edges.clear();
    }
    void AddEdge(int from, int to, int cap)
    {
        edges.emplace_back(from, to, cap, 0);
        edges.emplace_back(to, from, 0, 0); //反向弧
        m = edges.size();
        G[from].push_back(m - 2);
        G[to].push_back(m - 1);
    }
    int Maxflow(int s, int t)
    {
        int flow = 0;
        for (;;)
        {
            memset(a, 0, sizeof(a));
            queue<int> q;
            q.push(s);
            a[s] = INF;
            while (!q.empty())
            {
                int x = q.front();
                q.pop();
                for (int i = 0; i < G[x].size(); i++)
                {
                    Edge& e = edges[G[x][i]];
                    if (!a[e.to] && e.cap > e.flow)
                    {
                        p[e.to] = G[x][i];
                        a[e.to] = min(a[x], e.cap - e.flow);
                        q.push(e.to);
                    }
                }
                if (a[t]) break;
            }
            if (!a[t]) break;
            for (int u = t; u != s; u = edges[p[u]].from)
            {
                edges[p[u]].flow += a[t];
                edges[p[u] ^ 1].flow -= a[t];
            }
            flow += a[t];
        }
        return flow;
    }
};
\end{lstlisting}
\subsubsection{Dinic}
\begin{lstlisting}
const int maxn = "Edit";
struct Dinic
{
    int n, m, s, t;      //结点数,边数(包括反向弧),源点编号和汇点编号
    vector<Edge> edges;  //边表。edge[e]和edge[e^1]互为反向弧
    vector<int> G[maxn]; //邻接表,G[i][j]表示节点i的第j条边在e数组中的序号
    bool vis[maxn];      //BFS使用
    int d[maxn];         //从起点到i的距离
    int cur[maxn];       //当前弧下标
    void init(int n)
    {
        this->n = n;
        for (int i = 0; i < n; i++) G[i].clear();
        edges.clear();
    }
    void AddEdge(int from, int to, int cap)
    {
        edges.emplace_back(from, to, cap, 0);
        edges.emplace_back(to, from, 0, 0);
        m = edges.size();
        G[from].push_back(m - 2);
        G[to].push_back(m - 1);
    }
    bool BFS()
    {
        memset(vis, 0, sizeof(vis));
        memset(d, 0, sizeof(d));
        queue<int> q;
        q.push(s);
        d[s] = 0;
        vis[s] = 1;
        while (!q.empty())
        {
            int x = q.front();
            q.pop();
            for (int i = 0; i < G[x].size(); i++)
            {
                Edge& e = edges[G[x][i]];
                if (!vis[e.to] && e.cap > e.flow)
                {
                    vis[e.to] = 1;
                    d[e.to] = d[x] + 1;
                    q.push(e.to);
                }
            }
        }
        return vis[t];
    }
    int DFS(int x, int a)
    {
        if (x == t || a == 0) return a;
        int flow = 0, f;
        for (int& i = cur[x]; i < G[x].size(); i++)
        { //从上次考虑的弧
            Edge& e = edges[G[x][i]];
            if (d[x] + 1 == d[e.to] && (f = DFS(e.to, min(a, e.cap - e.flow))) > 0)
            {
                e.flow += f;
                edges[G[x][i] ^ 1].flow -= f;
                flow += f;
                a -= f;
                if (a == 0) break;
            }
        }
        return flow;
    }
    int Maxflow(int s, int t)
    {
        this->s = s, this->t = t;
        int flow = 0;
        while (BFS())
        {
            memset(cur, 0, sizeof(cur));
            flow += DFS(s, INF);
        }
        return flow;
    }
};
\end{lstlisting}
\subsubsection{ISAP}
\begin{lstlisting}
const int maxn = "Edit";
struct ISAP
{
    int n, m, s, t;      //结点数,边数(包括反向弧),源点编号和汇点编号
    vector<Edge> edges;  //边表。edges[e]和edges[e^1]互为反向弧
    vector<int> G[maxn]; //邻接表,G[i][j]表示结点i的第j条边在e数组中的序号
    bool vis[maxn];      //BFS使用
    int d[maxn];         //起点到i的距离
    int cur[maxn];       //当前弧下标
    int p[maxn];         //可增广路上的一条弧
    int num[maxn];       //距离标号计数
    void init(int n)
    {
        this->n = n;
        for (int i = 0; i < n; i++) G[i].clear();
        edges.clear();
    }
    void AddEdge(int from, int to, int cap)
    {
        edges.emplace_back(from, to, cap, 0);
        edges.emplace_back(to, from, 0, 0);
        int m = edges.size();
        G[from].push_back(m - 2);
        G[to].push_back(m - 1);
    }
    int Augumemt()
    {
        int x = t, a = INF;
        while (x != s)
        {
            Edge& e = edges[p[x]];
            a = min(a, e.cap - e.flow);
            x = edges[p[x]].from;
        }
        x = t;
        while (x != s)
        {
            edges[p[x]].flow += a;
            edges[p[x] ^ 1].flow -= a;
            x = edges[p[x]].from;
        }
        return a;
    }
    void BFS()
    {
        memset(vis, 0, sizeof(vis));
        memset(d, 0, sizeof(d));
        queue<int> q;
        q.push(t);
        d[t] = 0;
        vis[t] = 1;
        while (!q.empty())
        {
            int x = q.front();
            q.pop();
            int len = G[x].size();
            for (int i = 0; i < len; i++)
            {
                Edge& e = edges[G[x][i] ^ 1];
                if (!vis[e.from] && e.cap > e.flow)
                {
                    vis[e.from] = 1;
                    d[e.from] = d[x] + 1;
                    q.push(e.from);
                }
            }
        }
    }
    int Maxflow(int s, int t)
    {
        this->s = s;
        this->t = t;
        int flow = 0;
        BFS();
        memset(num, 0, sizeof(num));
        for (int i = 0; i < n; i++)
            if (d[i] < INF) num[d[i]]++;
        int x = s;
        memset(cur, 0, sizeof(cur));
        while (d[s] < n)
        {
            if (x == t)
            {
                flow += Augumemt();
                x = s;
            }
            int ok = 0;
            for (int i = cur[x]; i < G[x].size(); i++)
            {
                Edge& e = edges[G[x][i]];
                if (e.cap > e.flow && d[x] == d[e.to] + 1)
                {
                    ok = 1;
                    p[e.to] = G[x][i];
                    cur[x] = i;
                    x = e.to;
                    break;
                }
            }
            if (!ok) //Retreat
            {
                int m = n - 1;
                for (int i = 0; i < G[x].size(); i++)
                {
                    Edge& e = edges[G[x][i]];
                    if (e.cap > e.flow) m = min(m, d[e.to]);
                }
                if (--num[d[x]] == 0) break; //gap优化
                num[d[x] = m + 1]++;
                cur[x] = 0;
                if (x != s) x = edges[p[x]].from;
            }
        }
        return flow;
    }
};
\end{lstlisting}
\subsubsection{MinCost MaxFlow}
\begin{lstlisting}
const int maxn = "Edit";
struct MCMF
{
    int n, m;
    vector<Edge> edges;
    vector<int> G[maxn];
    int inq[maxn]; //是否在队列中
    int d[maxn];   //bellmanford
    int p[maxn];   //上一条弧
    int a[maxn];   //可改进量
    void init(int n)
    {
        this->n = n;
        for (int i = 0; i < n; i++) G[i].clear();
        edges.clear();
    }
    void AddEdge(int from, int to, int cap, int cost)
    {
        edges.emplace_back(from, to, cap, 0, cost);
        edges.emplace_back(to, from, 0, 0, -cost);
        m = edges.size();
        G[from].push_back(m - 2);
        G[to].push_back(m - 1);
    }
    bool BellmanFord(int s, int t, int& flow, ll& cost)
    {
        for (int i = 0; i < n; i++) d[i] = INF;
        memset(inq, 0, sizeof(inq));
        d[s] = 0;
        inq[s] = 1;
        p[s] = 0;
        a[s] = INF;
        queue<int> q;
        q.push(s);
        while (!q.empty())
        {
            int u = q.front();
            q.pop();
            inq[u] = 0;
            for (int i = 0; i < G[u].size(); i++)
            {
                Edge& e = edges[G[u][i]];
                if (e.cap > e.flow && d[e.to] > d[u] + e.cost)
                {
                    d[e.to] = d[u] + e.cost;
                    p[e.to] = G[u][i];
                    a[e.to] = min(a[u], e.cap - e.flow);
                    if (!inq[e.to])
                    {
                        q.push(e.to);
                        inq[e.to] = 1;
                    }
                }
            }
        }
        if (d[t] == INF) return false; // 当没有可增广的路时退出
        flow += a[t];
        cost += (ll)d[t] * (ll)a[t];
        for (int u = t; u != s; u = edges[p[u]].from)
        {
            edges[p[u]].flow += a[t];
            edges[p[u] ^ 1].flow -= a[t];
        }
        return true;
    }
    int MincostMaxflow(int s, int t, ll& cost)
    {
        int flow = 0;
        cost = 0;
        while (BellmanFord(s, t, flow, cost));
        return flow;
    }
};
\end{lstlisting}
\subsubsection{Upper-Lower Bound}
\textbf{上下界网络流建图方法}
\subsubsection*{记号说明}
\begin{itemize}
\item $f(u,v)$表示$u \rightarrow v$的实际流量
\item $b(u,v)$表示$u \rightarrow v$的流量下界
\item $c(u,v)$表示$u \rightarrow v$的流量上界
\end{itemize}

\subsubsection*{无源汇可行流}
\paragraph{建图}
\begin{itemize}
\item 新建附加源点$S$和$T$
\item 原图中的边$u \rightarrow v$,限制为$[b,c]$,建边$u \rightarrow v$,容量为$c-b$
\item 记$d(i) = \sum b(u,i) - \sum b(i,v) $
\item 若$d(i)>0$,建边$S \rightarrow i$,流量为$d(i)$
\item 若$d(i)<0$,建边$i \rightarrow T$,流量为$-d(i)$
\end{itemize}

\paragraph{求解}
\begin{itemize}
\item 跑$S \rightarrow T$的最大流,如果满流,则原图存在可行流。
\item 此时,原图中每一条边的流量为新图中对应边的流量加上这条边的下界。
\end{itemize}

\subsubsection*{有源汇可行流}
\paragraph{建图}
\begin{itemize}
\item 在原图中建边$t \rightarrow s$,流量限制为$[0,+\infty)$,这样就改造成了无源汇的网络流图。
\item 之后就可以像求解无源汇可行流一样建图了。
\end{itemize}

\paragraph{求解} 同无源汇可行流

\subsubsection*{有源汇最大流}
\paragraph{建图} 同有源汇可行流
\paragraph{求解}
\begin{itemize}
\item 先跑一遍$S \rightarrow T$的最大流,求出可行流
\item 记此时$\sum f(s,i) = sum_1$
\item 将$t \rightarrow s$这条边拆掉,在新图上跑$s \rightarrow t$的最大流
\item 记此时$\sum f(s,i) = sum_2$
\item 最终答案即为$sum_1 + sum_2$
\end{itemize}

\subsubsection*{有源汇最小流}
\paragraph{建图} 同无源汇可行流
\paragraph{求解}
\begin{itemize}
\item 求$S \rightarrow T$最大流
\item 建边$t \rightarrow s$,容量为$+\infty$
\item 再跑一遍$S \rightarrow T$的最大流,答案即为$f(t,s)$
\end{itemize}

\textbf{有源汇的最大流和最小流也可以通过二分答案求得,\\即二分$t \rightarrow s$的下界(最大流)和上界(最小流)复杂度多了个$O(\log n)$这里不再赘述。}

\paragraph{蓝书上的做法}
\begin{itemize}
\item 先用无源汇可行流建图的方法求出可行流,然后用传统$s-t$增广路算法即可得到最大流。把$t$看成源点,$s$看成汇点后求出的$t-s$最大流就是最小流。
\item 注意:原先每条弧$u \rightarrow v$的反向弧容量为$0$,而在有容量下界的情形中,反向弧的容量应该等于流量下界。
\end{itemize}

\subsubsection*{有源汇费用流}
\paragraph{建图}
\begin{itemize}
\item 新建附加源点$S$和$T$
\item 原图中的边$u \rightarrow v$,限制为$[b,c]$,费用为$cost$,建边$u \rightarrow v$,容量为$c-b$,费用为$cost$
\item 记$d(i) = \sum b(u,i) - \sum b(i,v) $
\item 若$d(i)>0$,建边$S \rightarrow i$,流量为$d(i)$,\textbf{费用为$0$}
\item 若$d(i)<0$,建边$i \rightarrow T$,流量为$-d(i)$,\textbf{费用为$0$}
\item 建边$t \rightarrow s$,流量为$+\infty$,费用为$0$。
\end{itemize}

\paragraph{求解}
\begin{itemize}
\item 跑$S \rightarrow T$的最小费用最大流
\item 答案为求出的费用加上原图中边的下界乘以边的费用
\end{itemize}
\clearpage\section{Computational Geometry}
\subsection{Basic Function}
\begin{lstlisting}
#define zero(x) ((fabs(x) < eps ? 1 : 0))
#define sgn(x) (fabs(x) < eps ? 0 : ((x) < 0 ? -1 : 1))

struct point
{
    double x, y;
    point(double a = 0, double b = 0) { x = a, y = b; }
    point operator-(const point& b) const { return point(x - b.x, y - b.y); }
    point operator+(const point& b) const { return point(x + b.x, y + b.y); }
    // 两点是否重合
    bool operator==(point& b) { return zero(x - b.x) && zero(y - b.y); }
    // 点积(以原点为基准)
    double operator*(const point& b) const { return x * b.x + y * b.y; }
    // 叉积(以原点为基准)
    double operator^(const point& b) const { return x * b.y - y * b.x; }
    // 绕P点逆时针旋转a弧度后的点
    point rotate(point b, double a)
    {
        double dx, dy;
        (*this - b).split(dx, dy);
        double tx = dx * cos(a) - dy * sin(a);
        double ty = dx * sin(a) + dy * cos(a);
        return point(tx, ty) + b;
    }
    // 点坐标分别赋值到a和b
    void split(double& a, double& b) { a = x, b = y; }
};
struct line
{
    point s, e;
    line() {}
    line(point ss, point ee) { s = ss, e = ee; }
};
\end{lstlisting}
\subsection{Position}
\subsubsection{Point-Point}
\begin{lstlisting}
double dist(point a, point b) { return sqrt((a - b) * (a - b)); }
\end{lstlisting}
\subsubsection{Line-Line}
\begin{lstlisting}
// <0, *> 表示重合; <1, *> 表示平行; <2, P> 表示交点是P;
pair<int, point> spoint(line l1, line l2)
{
    point res = l1.s;
    if (sgn((l1.s - l1.e) ^ (l2.s - l2.e)) == 0)
        return {sgn((l1.s - l2.e) ^ (l2.s - l2.e)) != 0, res};
    double t = ((l1.s - l2.s) ^ (l2.s - l2.e)) / ((l1.s - l1.e) ^ (l2.s - l2.e));
    res.x += (l1.e.x - l1.s.x) * t;
    res.y += (l1.e.y - l1.s.y) * t;
    return {2, res};
}
\end{lstlisting}
\subsubsection{Segment-Segment}
\begin{lstlisting}
bool segxseg(line l1, line l2)
{
    return
        max(l1.s.x, l1.e.x) >= min(l2.s.x, l2.e.x) &&
        max(l2.s.x, l2.e.x) >= min(l1.s.x, l1.e.x) &&
        max(l1.s.y, l1.e.y) >= min(l2.s.y, l2.e.y) &&
        max(l2.s.y, l2.e.y) >= min(l1.s.y, l1.e.y) &&
        sgn((l2.s - l1.e) ^ (l1.s - l1.e)) * sgn((l2.e-l1.e) ^ (l1.s - l1.e)) <= 0 &&
        sgn((l1.s - l2.e) ^ (l2.s - l2.e)) * sgn((l1.e-l2.e) ^ (l2.s - l2.e)) <= 0;
}
\end{lstlisting}
\subsubsection{Line-Segment}
\begin{lstlisting}
//l1是直线,l2是线段
bool segxline(line l1, line l2)
{
    return sgn((l2.s - l1.e) ^ (l1.s - l1.e)) * sgn((l2.e - l1.e) ^ (l1.s - l1.e)) <= 0;
}
\end{lstlisting}
\subsubsection{Point-Line}
\begin{lstlisting}
double pointtoline(point p, line l)
{
    point res;
    double t = ((p - l.s) * (l.e - l.s)) / ((l.e - l.s) * (l.e - l.s));
    res.x = l.s.x + (l.e.x - l.s.x) * t, res.y = l.s.y + (l.e.y - l.s.y) * t;
    return dist(p, res);
}
\end{lstlisting}
\subsubsection{Point-Segment}
\begin{lstlisting}
double pointtosegment(point p, line l)
{
    point res;
    double t = ((p - l.s) * (l.e - l.s)) / ((l.e - l.s) * (l.e - l.s));
    if (t >= 0 && t <= 1)
        res.x = l.s.x + (l.e.x - l.s.x) * t, res.y = l.s.y + (l.e.y - l.s.y) * t;
    else
        res = dist(p, l.s) < dist(p, l.e) ? l.s : l.e;
    return dist(p, res);
}
\end{lstlisting}
\subsubsection{Point on Segment}
\begin{lstlisting}
bool PointOnSeg(point p, line l)
{
    return
        sgn((l.s - p) ^ (l.e-p)) == 0 &&
        sgn((p.x - l.s.x) * (p.x - l.e.x)) <= 0 &&
        sgn((p.y - l.s.y) * (p.y - l.e.y)) <= 0;
}
\end{lstlisting}
\subsection{Polygon}
\subsubsection{Area}
\begin{lstlisting}
double area(point p[], int n)
{
    double res = 0;
    for (int i = 0; i < n; i++) res += (p[i] ^ p[(i + 1) % n]) / 2;
    return fabs(res);
}
\end{lstlisting}
\subsubsection{Point in Convex}
\begin{lstlisting}
// 点形成一个凸包, 而且按逆时针排序(如果是顺时针把里面的<0改为>0)
// 点的编号 : [0,n)
// -1 : 点在凸多边形外
// 0  : 点在凸多边形边界上
// 1  : 点在凸多边形内
int PointInConvex(point a, point p[], int n)
{
    for (int i = 0; i < n; i++)
        if (sgn((p[i] - a) ^ (p[(i + 1) % n] - a)) < 0)
            return -1;
        else if (PointOnSeg(a, line(p[i], p[(i + 1) % n])))
            return 0;
    return 1;
}
\end{lstlisting}
\subsubsection{Point in Polygon}
\begin{lstlisting}
// 射线法,poly[]的顶点数要大于等于3,点的编号0~n-1
// -1 : 点在凸多边形外
// 0  : 点在凸多边形边界上
// 1  : 点在凸多边形内
int PointInPoly(point p, point poly[], int n)
{
    int cnt;
    line ray, side;
    cnt = 0;
    ray.s = p;
    ray.e.y = p.y;
    ray.e.x = -100000000000.0; // -INF,注意取值防止越界
    for (int i = 0; i < n; i++)
    {
        side.s = poly[i], side.e = poly[(i + 1) % n];
        if (PointOnSeg(p, side)) return 0;
        //如果平行轴则不考虑
        if (sgn(side.s.y - side.e.y) == 0)
            continue;
        if (PointOnSeg(sid e.s, r ay))
            cnt += (sgn(side.s.y - side.e.y) > 0);
        else if (PointOnSeg(side.e, ray))
            cnt += (sgn(side.e.y - side.s.y) > 0);
        else if (segxseg(ray, side))
            cnt++;
    }
    return cnt % 2 == 1 ? 1 : -1;
}
\end{lstlisting}
\subsubsection{Judge Convex}
\begin{lstlisting}
//点可以是顺时针给出也可以是逆时针给出
//点的编号1~n-1
bool isconvex(point poly[], int n)
{
    bool s[3];
    memset(s, 0, sizeof(s));
    for (int i = 0; i < n; i++)
    {
        s[sgn((poly[(i + 1) % n] - poly[i]) ^ (poly[(i + 2) % n] - poly[i])) + 1] = 1;
        if (s[0] && s[2]) return 0;
    }
    return 1;
}
\end{lstlisting}
\subsection{Integer Points}
\subsubsection{On Segment}
\begin{lstlisting}
int OnSegment(line l) { return __gcd(fabs(l.s.x - l.e.x), fabs(l.s.y - l.e.y)) + 1; }
\end{lstlisting}
\subsubsection{On Polygon Edge}
\begin{lstlisting}
int OnEdge(point p[], int n)
{
    int i, ret = 0;
    for (i = 0; i < n; i++)
        ret += __gcd(fabs(p[i].x - p[(i + 1) % n].x), fabs(p[i].y - p[(i + 1) % n].y));
    return ret;
}
\end{lstlisting}
\subsubsection{Inside Polygon}
\begin{lstlisting}
int InSide(point p[], int n)
{
    int i, area = 0;
    for (i = 0; i < n; i++)
        area += p[(i + 1) % n].y * (p[i].x - p[(i + 2) % n].x);
    return (fabs(area) - OnEdge(p, n)) / 2 + 1;
}
\end{lstlisting}
\subsection{Circle}
\subsubsection{Circumcenter}
\begin{lstlisting}
point waixin(point a, point b, point c)
{
    double a1 = b.x - a.x, b1 = b.y - a.y, c1 = (a1 * a1 + b1 * b1) / 2;
    double a2 = c.x - a.x, b2 = c.y - a.y, c2 = (a2 * a2 + b2 * b2) / 2;
    double d = a1 * b2 - a2 * b1;
    return point(a.x + (c1 * b2 - c2 * b1) / d, a.y + (a1 * c2 - a2 * c1) / d);
}
\end{lstlisting}
\subsection{RuJia Liu's}
\subsubsection{Point}
\begin{lstlisting}
struct Point
{
    double x, y;
    Point(double x = 0, double y = 0) : x(x), y(y) {}
};

typedef Point Vector;

//向量+向量=向量,点+向量=点
Vector operator+(Vector A, Vector B) { return Vector(A.x + B.x, A.y + B.y); }
//点-点=向量
Vector operator-(Point A, Point B) { return Vector(A.x - B.x, A.y - B.y); }
//向量*数=向量
Vector operator*(Vector A, double p) { return Vector(A.x * p, A.y * p); }
//向量/数=向量
Vector operator/(Vector A, double p) { return Vector(A.x / p, A.y / p); }

bool operator<(const Point& a, const Point& b)
{
    return a.x < b.x || (a.x == b.x && a.y < b.y);
}

const double eps = 1e-10;
double dcmp(double x)
{
    if (fabs(x) < eps)
        return 0;
    else
        return x < 0 ? -1 : 1;
}

bool operator==(const Point& a, const Point& b)
{
    return dcmp(a.x - b.x) == 0 && dcmp(a.y - b.y) == 0;
}

/*
 * 基本运算:
 * 点积
 * 叉积
 * 向量旋转
 */
double Dot(Vector A, Vector B) { return A.x * B.x + A.y * B.y; }
double Length(Vector A) { return sqrt(Dot(A, A)); }
double Angle(Vector A, Vector B) { return acos(Dot(A, B) / Length(A) / Length(B)); }

double Cross(Vector A, Vector B) { return A.x * B.y - A.y * B.x; }
double Area2(Point A, Point B, Point C) { return Cross(B - A, C - A); }

//rad是弧度
Vector Rotate(Vector A, double rad)
{
    return Vector(A.x * cos(rad) - A.y * sin(rad),
                  A.x * sin(rad) + A.y * cos(rad));
}

//调用前请确保A不是零向量
Vector Normal(Vector A)
{
    double L = Length(A);
    return Vector(-A.y / L, A.x / L);
}

/*
 * 点和直线:
 * 两直线交点
 * 点到直线的距离
 * 点到线段的距离
 * 点在直线上的投影
 * 线段相交判定
 * 点在线段上判定
 */

//调用前保证两条直线P+tv和Q+tw有唯一交点。当且仅当Cross(v, w)非0
Point GetLineIntersection(Point P, Vector v, Point Q, Vector w)
{
    Vector u = P - Q;
    double t = Cross(w, u) / Cross(v, w);
    return P + v * t;
}

double DistanceToLine(Point P, Point A, Point B)
{
    Vector v1 = B - A, v2 = P - A;
    return fabs(Cross(v1, v2)) / Length(v1); //如果不取绝对值,得到的是有向距离
}

double DistanceToSegment(Point P, Point A, Point B)
{
    if (A == B) return Length(P - A);
    Vector v1 = B - A, v2 = P - A, v3 = P - B;
    if (dcmp(Dot(v1, v2)) < 0) return Length(v2);
    if (dcmp(Dot(v1, v3)) > 0) return Length(v3);
    return fabs(Cross(v1, v2)) / Length(v1);
}

Point GetLineProjection(Point P, Point A, Point B)
{
    Vector v = B - A;
    return A + v * (Dot(v, P - A) / Dot(v, v));
}

bool SegmentProperIntersection(Point a1, Point a2, Point b1, Point b2)
{
    double c1 = Cross(a2 - a1, b1 - a1), c2 = Cross(a2 - a1, b2 - b1),
           c3 = Cross(b2 - b1, a1 - b1), c4 = Cross(b2 - b1, a2 - b1);
    return dcmp(c1) * dcmp(c2) < 0 && dcmp(c3) * dcmp(c4) < 0;
}

bool OnSegment(Point p, Point a1, Point a2)
{
    return dcmp(Cross(a1 - p, a2 - p)) == 0 && dcmp(Dot(a1 - p, a2 - p)) < 0;
}\end{lstlisting}
\subsubsection{Circle}
\begin{lstlisting}
struct Line
{
    Point p;    //直线上任意一点
    Vector v;   //方向向量。它的左边就是对应的半平面
    double ang; //极角。即从x正半轴旋转到向量v所需要的角(弧度)
    Line() {}
    Line(Point p, Vector v) : p(p), v(v) { ang = atan2(v.y, v.x); }
    bool operator<(const Line& L) const // 排序用的比较运算符
    {
        return ang < L.ang;
    }
    Point point(double t) { return p + v * t; }
};

struct Circle
{
    Point c;
    double r;
    Circle(Point c, double r) : c(c), r(r) {}
    Point point(double a) { return c.x + cos(a) * r, c.y + sin(a) * r; }
};

int getLineCircleIntersection(Line L, Circle C, double& t1, double& t2, vector<Point>& sol)
{
    double a = L.v.x, b = L.p.x - C.c.x, c = L.v.y, d = L.p.y - C.c.y;
    double e = a * a + c * c, f = 2 * (a * b + c * d), g = b * b + d * d - C.r * C.r;
    double delta = f * f - 4 * e * g; //判别式
    if (dcmp(delta) < 0) return 0;    //相离
    if (dcmp(delta) == 0)             //相切
    {
        t1 = t2 = -f / (2 * e);
        sol.push_back(L.point(t1));
        return 1;
    }
    //相交
    t1 = (-f - sqrt(delta)) / (2 * e);
    t2 = (-f + sqrt(delta)) / (2 * e);
    sol.push_back(t1);
    sol.push_back(t2);
    return 2;
}

double angle(Vector v) { return atan2(v.y, v.x); }

int getCircleCircleIntersection(Circle C1, Circle C2, vector<Point>& sol)
{
    double d = Length(C1.c - C2.c);
    if (dcmp(d) == 0)
    {
        if (dcmp(C1.r - C2.r) == 0) return -1; //两圆重合
        return 0;
    }
    if (dcmp(C1.r + C2.r - d) < 0) return 0;       //内含
    if (dcmp(fabs(C1.r - C2.r) - d) > 0) return 0; //外离

    double a = angle(C2.c - C1.c); //向量C1C2的极角
    double da = acos((C1.r * C1.r + d * d - C2.r * C2.r) / (2 * C1.r * d));
    //C1C2到C1P1的角
    Point p1 = C1.point(a - da), p2 = C1.point(a + da);

    sol.push_back(p1);
    if (p1 == p2) return 1;
    sol.push_back(p2);
    return 2;
}

//过点p到圆C的切线,v[i]是第i条切线的向量,返回切线条数
int getTangents(Point p, Circle C, Vector* v)
{
    Vector u = C.c - p;
    double dist = Length(u);
    if (dist < C.r)
        return 0;
    else if (dcmp(dist - C.r) == 0)
    { //p在圆上,只有一条切线
        v[0] = Rotate(u, M_PI / 2);
        return 1;
    }
    else
    {
        double ang = asin(C.r / dist);
        v[0] = Rotate(u, -ang);
        v[1] = Rotate(u, +ang);
        return 2;
    }
}

//两圆的公切线
//返回切线的条数。-1表示无穷条切线。
//a[i]和b[i]分别是第i条切线在圆A和圆B上的切点
int getTangents(Circle A, Circle B, Point* a, Point* b)
{
    int cnt = 0;
    if (A.r < B.r)
    {
        swap(A, B);
        swap(a, b);
    }
    int d2 = (A.c.x - B.c.x) * (A.c.x - B.c.x) + (A.c.y - B.c.y) * (A.c.y - B.c.y);
    int rdiff = A.r - B.r;
    int rsum = A.r + B.r;
    if (d2 < rdiff * rdiff) return 0; //内含
    double base = atan2(B.c.y - A.c.y, B.c.x - A.c.x);
    if (d2 == 0 && A.r == B.r) return -1; //无限多条切线
    if (d2 == rdiff * rdiff)
    { //内切,一条切线
        a[cnt] = A.point(base);
        b[cnt] = B.point(base);
        cnt++;
        return 1;
    }
    //有外共切线
    double ang = acos((A.r - B.r) / sqrt(d2));
    a[cnt] = A.point(base + ang);
    b[cnt] = B.point(base + ang);
    cnt++;
    a[cnt] = A.point(base + ang);
    b[cnt] = B.point(base - ang);
    cnt++;
    if (d2 == rsum * rsum)
    {
        a[cnt] = A.point(base);
        b[cnt] = B.point(M_PI + base);
        cnt++;
    }
    else if (d2 > rsum * rsum)
    {
        double ang = acos((A.r + B.r) / sqrt(d2));
        a[cnt] = A.point(base + ang);
        b[cnt] = B.point(M_PI + base + ang);
        cnt++;
        a[cnt] = A.point(base - ang);
        b[cnt] = B.point(M_PI + base - ang);
        cnt++;
    }
    return cnt;
}

//三角形外接圆(三点保证不共线)
Circle CircumscribedCircle(Point p1, Point p2, Point p3)
{
    double Bx = p2.x - p1.x, By = p2.y - p1.y;
    double Cx = p3.x - p1.x, Cy = p3.y - p1.y;
    double D = 2 * (Bx * Cy - By * Cx);
    double cx = (Cy * (Bx * Bx + By * By) - By * (Cx * Cx + Cy * Cy)) / D + p1.x;
    double cy = (Bx * (Cx * Cx + Cy * Cy) - Cx * (Bx * Bx + By * By)) / D + p1.y;
    Point p = Point(cx, cy);
    return Circle(p, Length(p1 - p));
}

//三角形内切圆
Circle InscribedCircle(Point p1, Point p2, Point p3)
{
    double a = Length(p2 - p3);
    double b = Length(p3 - p1);
    double c = Length(p1 - p2);
    Point p = (p1 * a + p2 * b + p3 * c) / (a + b + c);
    return Circle(p, DistanceToLine(p, p1, p2));
}
\end{lstlisting}
\subsubsection{Polygon}
\begin{lstlisting}
typedef vector<Point> Polygon;
//多边形的有向面积
double PolygonArea(Polygon po)
{
    int n = po.size();
    double area = 0.0;
    for (int i = 1; i < n - 1; i++)
        area += Cross(po[i] - po[0], po[i + 1] - po[0]);
    return area / 2;
}

//点在多边形内判定
int isPointInPolygon(Point p, Polygon poly)
{
    int wn = 0; //绕数
    int n = poly.size();
    for (int i = 0; i < n; i++)
    {
        if (OnSegment(p, poly[i], poly[(i + 1) % n])) return -1; //边界上
        int k = dcmp(Cross(poly[(i + 1) % n] - poly[i], p - poly[i]));
        int d1 = dcmp(poly[i].y - p.y);
        int d2 = dcmp(poly[(i + 1) % n].y - p.y);
        if (k > 0 && d1 <= 0 && d2 > 0) wn++;
        if (k < 0 && d2 <= 0 && d1 > 0) wn--;
    }
    if (wn != 0) return 1; //内部
    return 0;              //外部
}

//凸包(Andrew算法)
//如果不希望在凸包的边上有输入点,把两个 <= 改成 <
//如果不介意点集被修改,可以改成传递引用
Polygon ConvexHull(vector<Point> p)
{
    sort(p.begin(), p.end());
    p.erase(unique(p.begin(), p.end()), p.end());
    int n = p.size(), m = 0;
    Polygon res(n + 1);
    for (int i = 0; i < n; i++)
    {
        while (m > 1 && Cross(res[m - 1] - res[m - 2], p[i] - res[m - 2]) <= 0) m--;
        res[m++] = p[i];
    }
    int k = m;
    for (int i = n - 2; i >= 0; i--)
    {
        while (m > k && Cross(res[m - 1] - res[m - 2], p[i] - res[m - 2]) <= 0) m--;
        res[m++] = p[i];
    }
    m -= n > 1;
    res.resize(m);
    return res;
}

//半平面交
vector<Point> HalfplaneIntersection(vector<Line>& L)
{
    int n = L.size();
    sort(L.begin(), L.end()); // 按极角排序

    int first, last;    // 双端队列的第一个元素和最后一个元素的下标
    vector<Point> p(n); // p[i]为q[i]和q[i+1]的交点
    vector<Line> q(n);  // 双端队列
    vector<Point> ans;  // 结果

    q[first = last = 0] = L[0]; // 双端队列初始化为只有一个半平面L[0]
    for (int i = 1; i < n; i++)
    {
        while (first < last && !OnLeft(L[i], p[last - 1])) last--;
        while (first < last && !OnLeft(L[i], p[first])) first++;
        q[++last] = L[i];
        if (fabs(Cross(q[last].v, q[last - 1].v)) < eps)
        { // 两向量平行且同向,取内侧的一个
            last--;
            if (OnLeft(q[last], L[i].p)) q[last] = L[i];
        }
        if (first < last) p[last - 1] = GetLineIntersection(q[last - 1], q[last]);
    }
    while (first < last && !OnLeft(q[first], p[last - 1])) last--; // 删除无用平面
    if (last - first <= 1) return vector<Point>();                 // 空集
    p[last] = GetLineIntersection(q[last], q[first]);              // 计算首尾两个半平面的交点

    return vector<Point>(q.begin() + first, q.begin() + last + 1);
}
\end{lstlisting}
\clearpage\section{Dynamic Programming}
\subsection{Subsequence}
\subsubsection{Max Sum}
\begin{lstlisting}
// 传入序列a和长度n,返回最大子序列和
int MaxSeqSum(int a[], int n)
{
    int rt = 0, cur = 0;
    for (int i = 0; i < n; i++)
        cur += a[i], rt = max(cur, rt), cur = max(0, cur);
    return rt;
}
\end{lstlisting}
\subsubsection{Longest Increase}
\begin{lstlisting}
// 序列下标从1开始,LIS()返回长度,序列存在lis[]中
const int N = "Edit";
int len, a[N], b[N], f[N];
int Find(int p, int l, int r)
{
    while (l <= r)
    {
        int mid = (l + r) >> 1;
        if (a[p] > b[mid])
            l = mid + 1;
        else
            r = mid - 1;
    }
    return f[p] = l;
}
int LIS(int lis[], int n)
{
    int len = 1;
    f[1] = 1, b[1] = a[1];
    for (int i = 2; i <= n; i++)
    {
        if (a[i] > b[len])
            b[++len] = a[i], f[i] = len;
        else
            b[Find(i, 1, len)] = a[i];
    }
    for (int i = n, t = len; i >= 1 && t >= 1; i--)
        if (f[i] == t) lis[--t] = a[i];
    return len;
}

// 简单写法(下标从0开始,只返回长度)
int dp[N];
int LIS(int a[], int n)
{
    memset(dp, 0x3f, sizeof(dp));
    for (int i = 0; i < n; i++) *lower_bound(dp, dp + n, a[i]) = a[i];
    return lower_bound(dp, dp + n, INF) - dp;
}
\end{lstlisting}
\subsubsection{Longest Common Increase}
\begin{lstlisting}
// 序列下标从1开始
int LCIS(int a[], int b[], int n, int m)
{
    memset(dp, 0, sizeof(dp));
    for (int i = 1; i <= n; i++)
    {
        int ma = 0;
        for (int j = 1; j <= m; j++)
        {
            dp[i][j] = dp[i - 1][j];
            if (a[i] > b[j]) ma = max(ma, dp[i - 1][j]);
            if (a[i] == b[j]) dp[i][j] = ma + 1;
        }
    }
    return *max_element(dp[n] + 1, dp[n] + 1 + m);
}
\end{lstlisting}
\subsection{Digit Statistics}
\begin{lstlisting}
int a[20];
ll dp[20][state];
ll dfs(int pos, /*state变量*/, bool lead /*前导零*/, bool limit /*数位上界变量*/)
{
    //递归边界,既然是按位枚举,最低位是0,那么pos==-1说明这个数枚举完了
    if (pos == -1) return 1;
    /*这里一般返回1,表示枚举的这个数是合法的,那么这里就需要在枚举时必须每一位都要满足题目条件,
    也就是说当前枚举到pos位,一定要保证前面已经枚举的数位是合法的。*/
    if (!limit && !lead && dp[pos][state] != -1) return dp[pos][state];
    /*常规写法都是在没有限制的条件记忆化,这里与下面记录状态是对应*/
    int up = limit ? a[pos] : 9; //根据limit判断枚举的上界up
    ll ans = 0;
    for (int i = 0; i <= up; i++) //枚举,然后把不同情况的个数加到ans就可以了
    {
        if () ...
        else if () ...
        ans += dfs(pos - 1, /*状态转移*/, lead && i == 0, limit && i == a[pos])
        //最后两个变量传参都是这样写的
        /*当前数位枚举的数是i,然后根据题目的约束条件分类讨论
        去计算不同情况下的个数,还有要根据state变量来保证i的合法性*/
    }
    //计算完,记录状态
    if (!limit && !lead) dp[pos][state] = ans;
    /*这里对应上面的记忆化,在一定条件下时记录,保证一致性,
    当然如果约束条件不需要考虑lead,这里就是lead就完全不用考虑了*/
    return ans;
}
ll solve(ll x)
{
    int pos = 0;
    do //把数位都分解出来
        a[pos++] = x % 10;
    while (x /= 10);
    return dfs(pos - 1 /*从最高位开始枚举*/, /*一系列状态 */, true, true);
    //刚开始最高位都是有限制并且有前导零的,显然比最高位还要高的一位视为0
}
\end{lstlisting}
\subsection{Slope Optimization}
\paragraph{问题}
设 $f(i) = \min(y[k] - s[i] \times x[k]), k \in [1,i-1]$, 现在要求出所有$f(i), i \in [1,n]$

考虑两个决策$j$和$k$,如果$j$比$k$优,则

$$y[j] - s[i] \times x[j] < y[k] - s[i] \times x[k]$$

化简得:

$$\frac{y_j - y_k}{x_j - x_k} < s_i$$

不等式左边是个斜率,我们把它设为$\text{slope}(j,k)$

我们可以维护一个单调递增的队列,为什么呢?

因为如果$\text{slope}(q[i-1],q[i]) > \text{slope}(q[i],q[i+1])$,那么当前者成立时,后者必定成立。 即$q[i]$决策优于$q[i-1]$决策时,$q[i+1]$必然优于$q[i]$,因此$q[i]$就没有存在的必要了。 所以我们要维护递增的队列。

那么每次的决策点$i$,都要满足
$$\begin{cases}
    \text{slope}(q[i-1],q[i]) < s[i] \\
    \text{slope}(q[i],q[i+1]) \ge s[i]
\end{cases}$$

一般情况去二分这个$i$即可。

如果$s[i]$是单调不降的,那么对于决策$j$和$k$($j < k$)来说,如果决策$k$优于决策$j$,那么对于$i \in [k+1,n]$,都存在决策$k$优于决策$j$, 因此决策$j$就可以舍弃了。 这样的话我们可以用单调队列进行优化,可以少个$\log$。

\paragraph{单调队列滑动窗口最大值}
\begin{lstlisting}
// k为滑动窗口的大小
deque<int> q;
for (int i = 0, j = 0; i + k <= d; i++)
{
    while (j < i + k)
    {
        while (!q.empty() && a[q.back()] < a[j]) q.pop_back();
        q.push_back(j++);
    }
    while (q.front() < i) q.pop_front();
    // a[q.front()]为当前滑动窗口的最大值
}
\end{lstlisting}\clearpage\section{Others}
\subsection{Matrix}
\subsubsection{Matrix FastPow}
\begin{lstlisting}
typedef vector<ll> vec;
typedef vector<vec> mat;
mat mul(mat& A, mat& B)
{
    mat C(A.size(), vec(B[0].size()));
    for (int i = 0; i < A.size(); i++)
        for (int k = 0; k < B.size(); k++)
            if (A[i][k]) // 对稀疏矩阵的优化
                for (int j = 0; j < B[0].size(); j++)
                    C[i][j] = (C[i][j] + A[i][k] * B[k][j]) % mod;
    return C;
}
mat Pow(mat A, ll n)
{
    mat B(A.size(), vec(A.size()));
    for (int i = 0; i < A.size(); i++) B[i][i] = 1;
    for (; n; n >>= 1, A = mul(A, A))
        if (n & 1) B = mul(B, A);
    return B;
}
\end{lstlisting}
\subsubsection{Gauss Elimination}
\begin{lstlisting}
void gauss()
{
    int now = 1, to;
    double t;
    for (int i = 1; i <= n; i++, now++)
    {
        /*for (to = now; !a[to][i] && to <= n; to++);
        //做除法时减小误差,可不写
        if (to != now)
            for (int j = 1; j <= n + 1; j++)
                swap(a[to][j], a[now][j]);*/
        t = a[now][i];
        for (int j = 1; j <= n + 1; j++) a[now][j] /= t;
        for (int j = 1; j <= n; j++)
            if (j != now)
            {
                t = a[j][i];
                for (int k = 1; k <= n + 1; k++) a[j][k] -= t * a[now][k];
            }
    }
}
\end{lstlisting}
\subsection{Tricks}
\subsubsection{Stack-Overflow}
\begin{lstlisting}
// 解决爆栈问题
#pragma comment(linker, "/STACK:1024000000,1024000000")
\end{lstlisting}
\subsubsection{Fast-Scanner}
\begin{lstlisting}
// 适用于正负整数
template <class T>
inline bool scan_d(T &ret)
{
    char c;
    int sgn;
    if (c = getchar(), c == EOF) return 0; //EOF
    while (c != '-' && (c < '0' || c > '9')) c = getchar();
    sgn = (c == '-') ? -1 : 1;
    ret = (c == '-') ? 0 : (c - '0');
    while (c = getchar(), c >= '0' && c <= '9') ret = ret * 10 + (c - '0');
    ret *= sgn;
    return 1;
}
inline void out(int x)
{
    if (x > 9) out(x / 10);
    putchar(x % 10 + '0');
}
\end{lstlisting}
\subsubsection{Strok-Sscanf}
\begin{lstlisting}
// 空格作为分隔输入,读取一行的整数
fgets(buf, BUFSIZE, stdin);
int v;
char *p = strtok(buf, " ");
while (p)
{
    sscanf(p, "%d", &v);
    p = strtok(NULL," ");
}
\end{lstlisting}
\subsection{Mo Algorithm}
\begin{lstlisting}
\end{lstlisting}
莫队算法,可以解决一类静态,离线区间查询问题。分成 $\sqrt{x}$ 块,分块排序。
\begin{lstlisting}
struct query { int L, R, id; };
void solve(query node[], int m)
{
    memset(ans, 0, sizeof(ans));
    sort(node, node + m, [](query a, query b) {
        return a.l / unit < b.l / unit
               || a.l / unit == b.l / unit && a.r < b.r;
    });
    int L = 1, R = 0;
    for (int i = 0; i < m; i++)
    {
        while (node[i].L < L) add(a[--L]);
        while (node[i].L > L) del(a[L++]);
        while (node[i].R < R) del(a[R--]);
        while (node[i].R > R) add(a[++R]);
        ans[node[i].id] = tmp;
    }
}
\end{lstlisting}
\subsection{BigNum}
\subsubsection{High-precision}
\begin{lstlisting}
// 加法 乘法 小于号 输出
struct bint
{
    int l;
    short int w[100];
    bint(int x = 0)
    {
        l = x == 0, memset(w, 0, sizeof(w));
        while (x) w[l++] = x % 10, x /= 10;
    }
    bool operator<(const bint& x) const
    {
        if (l != x.l) return l < x.l;
        int i = l - 1;
        while (i >= 0 && w[i] == x.w[i]) i--;
        return (i >= 0 && w[i] < x.w[i]);
    }
    bint operator+(const bint& x) const
    {
        bint ans;
        ans.l = l > x.l ? l : x.l;
        for (int i = 0; i < ans.l; i++)
        {
            ans.w[i] += w[i] + x.w[i];
            ans.w[i + 1] += ans.w[i] / 10;
            ans.w[i] = ans.w[i] % 10;
        }
        if (ans.w[ans.l] != 0) ans.l++;
        return ans;
    }
    bint operator*(const bint& x) const
    {
        bint res;
        int up, tmp;
        for (int i = 0; i < l; i++)
        {
            up = 0;
            for (int j = 0; j < x.l; j++)
            {
                tmp = w[i] * x.w[j] + res.w[i + j] + up;
                res.w[i + j] = tmp % 10;
                up = tmp / 10;
            }
            if (up != 0) res.w[i + x.l] = up;
        }
        res.l = l + x.l;
        while (res.w[res.l - 1] == 0 && res.l > 1) res.l--;
        return res;
    }
    void print()
    {
        for (int i = l - 1; ~i; i--) printf("%d", w[i]);
        puts("");
    }
};
\end{lstlisting}
\subsubsection{Complete High-precision}
\begin{lstlisting}
import java.math.BigInteger;\end{lstlisting}
\subsection{Misc}
\subsubsection{Standard Template Library}
\begin{lstlisting}
template <class InputIterator, class OutputIterator>
  OutputIterator copy (InputIterator first, InputIterator last, OutputIterator result);

template <class InputIterator1, class InputIterator2,
          class OutputIterator, class Compare>
  OutputIterator merge (InputIterator1 first1, InputIterator1 last1,
                        InputIterator2 first2, InputIterator2 last2,
                        OutputIterator result, Compare comp);

template <class InputIterator, class Function>
   Function for_each (InputIterator first, InputIterator last, Function fn);

template <class InputIterator, class OutputIterator, class UnaryOperation>
  OutputIterator transform (InputIterator first1, InputIterator last1,
                            OutputIterator result, UnaryOperation op);

template< class ForwardIterator, class T >
void iota( ForwardIterator first, ForwardIterator last, T value );

\end{lstlisting}
\subsubsection{Policy-Based Data Structures}
\paragraph{红黑树}
\subparagraph{声明/头文件}
\begin{lstlisting}
#include <ext/pb_ds/tree_policy.hpp>
#include <ext/pb_ds/assoc_container.hpp>
using namespace __gnu_pbds;
typedef tree<pt, null_type, less<pt>, rb_tree_tag, tree_order_statistics_node_update> rbtree;
\end{lstlisting}
\subparagraph{使用方法}
\begin{lstlisting}
pt                                 // 关键字类型
null_type                          // 无映射(低版本g++为null_mapped_type)
less<int>                          // 从小到大排序
rb_tree_tag                        // 红黑树(splay_tree_tag)
tree_order_statistics_node_update  // 结点更新
T.insert(val);                     // 插入
T.erase(iterator);                 // 删除
T.order_of_key();                  // 查找有多少数比它小
T.find_by_order(k);                // 有k个数比它小的数是多少
a.join(b);                         // b并入a 前提是两棵树的key的取值范围不相交
a.split(v, b);                     // key小于等于v的元素属于a,其余的属于b
T.lower_bound(x);                  // >=x的min的迭代器
T.upper_bound((x);                 // >x的min的迭代器
\end{lstlisting}
\subsubsection{Subset Enumeration}
\begin{lstlisting}
\end{lstlisting}
枚举真子集
\begin{lstlisting}
for (int s = (S - 1) & S; s; s = (s - 1) & S)
\end{lstlisting}
枚举大小为 $k$ 的子集
\begin{lstlisting}
void subset(int k, int n)
{
    int t = (1 << k) - 1;
    while (t < (1 << n))
    {
        // do something
        int x = t & -t, y = t + x;
        t = ((t & ~y) / x >> 1) | y;
    }
}\end{lstlisting}
\subsubsection{Date Magic}
\begin{lstlisting}
string dayOfWeek[] = {"Mo", "Tu", "We", "Th", "Fr", "Sa", "Su"};

// converts Gregorian date to integer (Julian day number)
int DateToInt(int m, int d, int y)
{
    return 1461 * (y + 4800 + (m - 14) / 12) / 4
           + 367 * (m - 2 - (m - 14) / 12 * 12) / 12
           - 3 * ((y + 4900 + (m - 14) / 12) / 100) / 4
           + d - 32075;
}

// converts integer (Julian day number) to Gregorian date: month/day/year
void IntToDate(int jd, int& m, int& d, int& y)
{
    int x, n, i, j;
    x = jd + 68569;
    n = 4 * x / 146097;
    x -= (146097 * n + 3) / 4;
    i = (4000 * (x + 1)) / 1461001;
    x -= 1461 * i / 4 - 31;
    j = 80 * x / 2447;
    d = x - 2447 * j / 80;
    x = j / 11;
    m = j + 2 - 12 * x;
    y = 100 * (n - 49) + i + x;
}

// converts integer (Julian day number) to day of week
string IntToDay(int jd) { return dayOfWeek[jd % 7]; }\end{lstlisting}
\subsection{Configuration}
\subsubsection{Vim}
\begin{lstlisting}
sy on
set et nu is sc ai hls cin udf
set bs=2 sw=4 scrolloff=999 sts=4 mouse=a cb=unnamed

colo evening
nnoremap 0 ^
map<c-y> mmggVG"+y`m
map<f5> :call Run()<cr>

func! Run()
    exec "w"
    exec "!g++ -std=c++11 -O2 % -o %<"
    exec "!time ./%<"
endfunc

inoremap ( ()<esc>i
inoremap { {}<esc>i
inoremap [ []<esc>i
inoremap {<cr> {<cr>}<esc>O
inoremap ) <c-r>=ClosePair(')')<cr>
inoremap } <c-r>=ClosePair('}')<cr>
inoremap ] <c-r>=ClosePair(']')<cr>

func ClosePair(char)
    if getline('.')[col('.') - 1] == a:char
        return "\<right>"
    else
        return a:char
    endif
endfunc

vnoremap \\ mm^o^<C-v>I//<ESC>`m
vnoremap \d mm^o^<C-v>ld<ESC>`m
\end{lstlisting}

\end{document}
